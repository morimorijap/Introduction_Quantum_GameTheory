\section{Quantum teleportation and pseudo-telepathy 量子テレポーテーションと疑似テレパシー}

アリスとボブは7光年離れており、絡み合った量子ビットのペアを共有しています。たとえば、
$ \ket{b_0} = \frac{1}{\sqrt{2}} ( \ket{00} + \ket{11})$です。 
アリスが自分のキュービットを測定し、それが状態$\ket{0}$にあることを検出した場合、ボブのキュービットも状態$\ket{0}$にあることが保証されます。 アリスが測定によって彼女のキュービットが状態$\ket{1}$にあることを検出した場合、ボブのキュービットも状態$\ket{1}$に検出されます。 つまり、アリスの測定値はボブのキュービットの状態に影響を与えます。 私たちが知る限り、ボーアチャネルを介したこの影響の伝達は瞬時に行われます。 距離の影響を受けたり、光速の影響を受けたりすることはありません。 遠隔作用で不気味なアクションです。 また、量子テレポーテーションの基礎でもあります。

\textbf{テレポーテーション}
対照的に、量子テレポーテーションプロトコル(参考文献2)は、ボーア(EPR)チャネルだけでなく古典的なチャネルも使用するため、瞬時には実行されません。
一方、量子状態はある場所で消え、別の場所で再び現れるため、テレポートされます。 従来のテレポーテーションプロトコルはこのように機能します。
アリスは未知の量子状態を持っています$\ket{\psi}$彼女はボブに送信したいと思っています。 彼女はこれを2つの部分で行います。エンタングルされたボーアチャネルと、いくつかのクラシックビットを送信するための追加のクラシックチャネルを使用します。
アリスとボブは、絡み合った粒子のペアを共有するために以前の取り決めをしました。今回はベル状態で$\ket{b_3}$:

\begin{equation}
\label{181}
\ket{b_3} = \frac{1}{\sqrt{2}}(\ket{01}-\ket{10}).
\end{equation}

アリスが送信しようとしている未知の状態は、未知の振幅$a, b, |a|^2 + |b|^2 = 1$で記述できます。

\begin{equation}
\label{182}
\ket{\psi} = a \ket{0} + b \ket{1}.
\end{equation}

3量子ビットシステムの初期状態を次のように書くことができます。


\begin{equation}
\label{183}
\ket{\psi} \otimes \ket{b_3} = ( a \ket{0} + b \ket{1}) \otimes \frac{1}{\sqrt{2}}(\ket{01}-\ket{10})
\end{equation}

\begin{equation}
\label{184}
= \frac{a}{\sqrt{2}} \ket{001} -\frac{a}{\sqrt{2}} \ket{010} + \frac{b}{\sqrt{2}} \ket{101} -\frac{b}{\sqrt{2}} \ket{110} .
\end{equation}

明らかになる理由から、この状態をベル基底の観点から書き直したいと思います。 これを行うには、各ベル状態の乗数を見つけるために、$\ket{psi} \otimes \ket{b_3}$と各ベルベクトルの内積を取ります。 式(184)の左端の2つのキュービットを持つ内積をとることに注意してください。 これらのキュービットはアリスの管理下にあります。

\begin{equation}
\label{185}
\braket{b_0 | (\ket{\psi} \otimes \ket{b_3})}
=
+ \frac{a}{\sqrt{2}}\ket{1} - \frac{b}{\sqrt{2}}\ket{0} 
\end{equation}

\begin{equation}
\label{186}
\braket{b_1 | (\ket{\psi} \otimes \ket{b_3})}
=
- \frac{a}{\sqrt{2}}\ket{1} + \frac{b}{\sqrt{2}}\ket{1} 
\end{equation}

\begin{equation}
\label{187}
\braket{b_2 | (\ket{\psi} \otimes \ket{b_3})}
=
+ \frac{a}{\sqrt{2}}\ket{1} + \frac{b}{\sqrt{2}}\ket{1} 
\end{equation}

\begin{equation}
\label{188}
\braket{b_3 | (\ket{\psi} \otimes \ket{b_3})}
=
- \frac{a}{\sqrt{2}}\ket{0} - \frac{b}{\sqrt{2}}\ket{1}.
\end{equation}

これらの残差状態乗数を使用して、ベル基底の観点から状態$\ket{psi} \otimes \ket{b_3}$を書くことができます。

\begin{equation}
\label{189}
\ket{\psi} \otimes \ket{b_3}
=
\frac{1}{\sqrt{2}}
[ 
\left( \begin{array}{c}
 -b \\
 +a
\end{array} \right)  \ket{b_0}
+
\frac{1}{\sqrt{2}}
\left( \begin{array}{c}
 -a \\
 +a
\end{array} \right)  \ket{b_1}
+
\frac{1}{\sqrt{2}}
\left( \begin{array}{c}
 +b \\
 +a
\end{array} \right)  \ket{b_2}
+
\frac{1}{\sqrt{2}}
\left( \begin{array}{c}
 -a \\
 -b
\end{array} \right)  \ket{b_3} ].
\end{equation}

次に、最後の方程式を2×2行列で書き直してみましょう。

\begin{equation}
\label{190}
\ket{\psi} \otimes \ket{b_3}
=
\frac{1}{\sqrt{2}}
[ 
\left( \begin{array}{cc}
 0 & -1 \\
 1 & 0
\end{array} \right)  
\left( \begin{array}{c}
 a \\
 b
\end{array} \right) 
\ket{b_0}
+
\left( \begin{array}{cc}
 -1 & 0 \\
 0 & 1
\end{array} \right)  
\left( \begin{array}{c}
 a \\
 b
\end{array} \right)  \ket{b_1}
+
\end{equation}

\begin{equation}
\label{191}
\left( \begin{array}{cc}
 0 & 1 \\
 1 & 0
\end{array} \right)  
\left( \begin{array}{c}
 a \\
 b
\end{array} \right) 
\ket{b_2}
+
\left( \begin{array}{cc}
 -1 & 0 \\
 0 & -1
\end{array} \right)  
\left( \begin{array}{c}
 a \\
 b
\end{array} \right)  \ket{b_3}
].
\end{equation}

これをパウリスピン行列の観点からもう一度書き直すことができます。

\begin{equation}
\label{192}
\ket{\psi} \otimes \ket{b_3}
=
\frac{1}{\sqrt{2}}
[ 
-i \sigma_y
\left( \begin{array}{c}
 a \\
 b
\end{array} \right) 
\ket{b_0}
-
\sigma_z
\left( \begin{array}{c}
 a \\
 b
\end{array} \right)  \ket{b_1}
+
\sigma_x
\left( \begin{array}{c}
 a \\
 b
\end{array} \right)  \ket{b_2}
-
\mathbf{1}
\left( \begin{array}{c}
 a \\
 b
\end{array} \right)  \ket{b_3} ].
\end{equation}

さて、彼女のキュービットをボブにテレポートするには、アリスは未知の状態$\ket{\psi}$を絡み合ったキュービットペアのメンバーと結合する必要があります。
これを行うために、彼女はこれら2つのキュービットのジョイント(フォンノイマン)測定を行います。これは、$\ket{\psi} \otimes \ket{b_3}$の左端の2つのキュービットを構成します。
アリスの測定は、彼女の2つのキュービットを4つのベル状態の1つに投影します。 これにより、未知の状態$\ket{\psi}$が破壊されます。
しかし、心配しないでください。 アリスの測定では、ボブのキュービットも次の4つの状態のいずれかになります。

\begin{equation}
\label{193}
\ket{\psi} \otimes \ket{b_3}
\to
\ket{b_0}
\longrightarrow
Bob's \ qubit
=
-i \sigma_y 
\left( \begin{array}{c}
 a \\
 b
\end{array} \right) 
\end{equation}


\begin{equation}
\label{194}
\ket{\psi} \otimes \ket{b_3}
\to
\ket{b_1}
\longrightarrow
Bob's \ qubit
=
- \sigma_z
\left( \begin{array}{c}
 a \\
 b
\end{array} \right) 
\end{equation}

\begin{equation}
\label{195}
\ket{\psi} \otimes \ket{b_3}
\to
\ket{b_2}
\longrightarrow
Bob's \ qubit
=
- \sigma_x
\left( \begin{array}{c}
 a \\
 b
\end{array} \right) 
\end{equation}

\begin{equation}
\label{196}
\ket{\psi} \otimes \ket{b_3}
\to
\ket{b_3}
\longrightarrow
Bob's \ qubit
=
- \mathbf{1}
\left( \begin{array}{c}
 a \\
 b
\end{array} \right) .
\end{equation}

次に、アリスは古典的なチャネルを介して、測定結果、つまり彼女が取得したベル状態をボブに送信します。
それから、ボブは対応するスピン演算子(それ自体の逆)をキュービットに適用して、
状態
$\ket{\psi} =  
\left( \begin{array}{c}
 a \\
 b
\end{array} \right) 
:
i \sigma_y \ for \ \ket{b_0},
- \sigma_z \ for \ \ket{b_1},
\sigma_x \ for \ \ket{b_2},
- \mathbf{1}  \ for \ \ket{b_3}
$
を回復します。
(実際には、$- \ket{\psi}$は$\ket{\psi}$と同じ状態であるため、全体的な符号[aとbの両方を等しく乗算する符号]は重要ではありません。たとえば、$\ket{\sigma_z}$または$\mathbf{1}$を乗算するだけで十分です。)

要約すると、アリスとボブは2つのキュービットのもつれ状態$\ket{\theta}$を共有します。 アリスは未知の状態$\ket{\psi}$をボブにテレポートしたいと思っています。 これを行うために、彼女は最初に2つのキュービットに基づいてベル基底で$\ket{\psi} \otimes \ket{\theta}$の測定を実行します。
(未知の状態、および絡み合った状態の彼女のキュービット)
彼女は、取得したベル状態の情報をボブに送信します。 ボブは対応するパウリスピン演算子を彼のキュービットに適用し、未知の状態$\ket{\psi}$を回復します。

\textbf{Pseudo − telepathy  疑似-テレパシー}
「エンタングルメントは、おそらく量子力学の最も非古典的な兆候です。 情報処理への多くの興味深いアプリケーションの中で、それを利用して、さまざまな分散計算タスクを処理するために必要な通信の量を減らすことができます。 コミュニケーションを完全になくすために使用できますか? リモートパーティ間で情報を通知することはできませんが、パーティが事前の絡み合いを共有していれば、通信を必要とせずに実行できる分散タスクがあります。これは疑似テレパシーの領域です。」(参考文献5)

$N$人のプレイヤー間の次の疑似テレパシーゲーム$\Gamma_N$を考えてみましょう。 2人以上のプレーヤーがいるため、それらをアリスとボブと呼ぶことはできません。そのため、すべてのプレーヤーを下付き文字のアリス:$A_1, A_2, \cdots, A_N$とします。 また、2つの関数$f$と$g$があり、それぞれが$N$キュービット入力を受け取ります。 ゲームには次の手順があります。

ステップ1:プレイヤーは交わり、戦略について話し合い、確率変数(古典的な設定)またはエンタングルメント(量子設定)を共有します。

ステップ2:プレーヤーは分離し、いかなる形式のコミュニケーションにも関与することは許可されていません。 各プレーヤー$A_i$には、単一の量子ビット入力$x_i$が与えられ、単一の量子ビット出力$y_i$を生成するように要求されます。 次の場合、プレイヤーは$+1$を獲得します

\begin{equation}
\label{197}
f(x_1, x_2, \cdots, x_N) = g(y_1, y_2, \cdots, y_N).
\end{equation}

そうでなければ、彼らはこの金額を失います。 関数fとgは次のように定義されます。 プレイヤーは、与えられたキュービットの合計が偶数であることが保証されます。$\sum_i x_i$は偶数です。
(これが何を意味するかを考えてください。$\sum_i x_i$が偶数の場合、2で割り切れます。したがって、$ \frac{1}{2} \sum_i x_i$は、奇数または偶数の整数です。奇数の場合、$ \frac{1}{2} \sum_i x_i \mod 2 = 1$です。偶数の場合、 次に$ \frac{1}{2} \sum_i x_i \mod 2 = 0$です。ただし、後者の場合は、$ \frac{1}{2} \sum_i x_i \mod 2$も2で割り切れるので、元の合計$\sum_i x_i$は4で割り切れます。)
入力ビット$\sum_i x_i$の合計が4で割り切れる場合に限り、プレーヤーは出力ビット$\sum_i y_i$の偶数の合計を生成するように求められます。したがって、$N$プレーヤーが勝つための基準は次のとおりです。

\begin{equation}
\label{198}
\sum_i y_i \mod 2 
=
\frac{1}{2} \sum_i x_i \mod 2.
\end{equation}

この方程式の左辺は$g$で、右辺は$f$です。 各プレーヤーが制御するのは1キュービットのみであり、他のプレーヤーとの通信は許可されていませんが、勝利は$N$キュービットのグローバル状態にのみ依存します。
ここで、$\mod 2$は2つの結果しか生成しないため、プレーヤー$i$が$y_i$の送信をランダム化した場合のプレーヤーへの期待されるペイオフは$0$であることに注意してください。
これは、プレイヤー間の協力の問題を浮き彫りにし、ゲームが任意の数Nのプレイヤーに拡張可能であるため、非常に優れたゲームです。

さて、驚くべきことは、ステップ$1$のように、プレーヤーが以前のエンタングルメントを共有することを許可されている場合、プレーヤーは常に$\Gamma_N$を獲得することです。
これをどのように行うかを確認するには、ベル状態$\ket{b_0}$と$\ket{b_2}$、アダマール変換$H$、およびカードゲームで導入されたユニタリ行列または回転行列をコンポーネントとして必要とします。ただし、ここでは次のように定義します。

\begin{equation}
\label{199}
U_{\frac{\pi}{2} }
=
\left( \begin{array}{cc}
 1 & 0 \\
 0 & e^{i \frac{\pi}{2}}
\end{array} \right) 
=
\left( \begin{array}{cc}
 1 & 0 \\
 0 & i
\end{array} \right),
\end{equation}

ここで、$\cos ( \frac{\pi}{2}) + i \sin( \frac{\pi}{2}) =i$であることを思い出してください。$U_{\frac{\pi}{2}} \ket{0} = \ket{0}$ですが、$U_{\frac{\pi}{2}} \ket{1} = \ket{1}$であることに注意してください。

$N$人のプレイヤーが絡み合ったベル状態を共有するため、後者はN-キュービットのベル状態である必要があります。 N-キュービットのベル状態を次の簡略化された形式で記述しましょう。

\begin{equation}
\label{200}
\ket{b_0^N}
=
\frac{1}{\sqrt{2}}(\ket{0^N} + \ket{1^N} )
\end{equation}

\begin{equation}
\label{201}
\ket{b_2^N}
=
\frac{1}{\sqrt{2}}(\ket{0^N} - \ket{1^N} ).
\end{equation}

最初の$N$キュービット状態$\ket{b_0^N}$ は、すべてのプレーヤーが共有することに同意するエンタングル状態です。 2番目の状態は、プレイの過程で進化する可能性があります。

ここで、$\ket{b_0^N}$ の単一キュービットで動作するユニタリ行列の効果を考えてみましょう。

\begin{equation}
\label{202}
U_{\frac{\pi}{2}}
\ket{b_2^N}
=
\frac{1}{\sqrt{2}}(\ket{0^N} + i \ket{1^N} ).
\end{equation}

$i$の累乗は$i, i^2 = -1, i^3 = -i, i^4 = 1$です。 したがって、$U_{\frac{\pi}{2}}$が2つのキュービットに適用される場合、$\ket{1^N}$の符号は$-1$になり、したがって$ \ket{b_0^N} \to \ket{b_2^N}$になります。 $4$キュービットに適用した場合、符号は変更されないため、$ \ket{b_0^N} \to \ket{b_2^N}$です。 したがって、$m$人のプレーヤーが個々のキュービットに$U_{\frac{\pi}{2}}$を適用する場合、$m = 0 \mod 4$の場合、初期状態$\ket{b_0^N}$ は変更されません。
$m = 2 \mod 4$の場合、$ \ket{b_0^N} \to \ket{b_2^N}$です。

エンタングル状態が$\ket{b_0^N}$ のときに各プレーヤーがアダマール行列をキュービットに適用すると、結果は1ビットの偶数であるすべての状態の重ね合わせになります。

\begin{equation}
\label{203}
(\mathbf{H} \otimes ^N)
\ket{b_2^N}
=
\frac{1}{\sqrt{2^{N-1}}}  \sum_{even \ bit \ y}^{2^{N-1}} \ket{y} .
\end{equation}

これは、合計の状態$\ket{y}$が偶数であることを意味するものではないことに注意してください。
たとえば、$\ket{101} = \ket{5}$は奇数ですが1ビットは偶数ですが、$\ket{100} = \ket{4}$は偶数ですが1ビットは奇数です。
N倍アダマール変換(ウォルシュ変換)がベル状態$\ket{b_0^N}$を偶数ビット数(1ビットの偶数を意味する)の重ね合わせに変換することを確認するには、表5の類似物である表13を検討してください。
マイナス記号は、奇数の1ビットの数値に表示されることに注意してください。

\begin{table}[htb]
\caption{初期量子ビットを使用したウォルシュ変換$\ket{111}$}
\centering
\begin{tabular}{|r|r|r|r|} \hline
$\ket{b}$ & $\ket{y}$ & $b \cdot y$ & $ (-1)^{b \cdot y}$ \\ \hline
$\ket{111}$ & $\ket{000}$ & 0 & 1 \\
$\ket{111}$ & $\ket{001}$ & 1 & -1 \\
$\ket{111}$ & $\ket{010}$ & 1 & -1 \\
$\ket{111}$ & $\ket{011}$ & 0 & 1 \\
$\ket{111}$ & $\ket{100}$ & 1 & -1 \\
$\ket{111}$ & $\ket{101}$ & 0 & 1 \\
$\ket{111}$ & $\ket{110}$ & 0 & 1 \\
$\ket{111}$ & $\ket{111}$ & 1 & -1 \\ \hline
\end{tabular}
\end{table} 

% 表XIII:初期量子ビットを使用したウォルシュ変換$\ket{111}$ \\

したがって、$( \mathbf{H} \otimes \mathbf{H} \otimes \mathbf{H})$を $\frac{1}{\sqrt{2}}(\ket{000} + \ket{111}) $に適用すると、$\frac{2}{\sqrt{2^4}}(\ket{0}+\ket{1}+\ket{2}+\ket{3}+\ket{4}+\ket{5}+\ket{6}+\ket{7} + \ket{0}-\ket{1}-\ket{2}+\ket{3}-\ket{4}+\ket{5}+\ket{6}-\ket{7} ) = \frac{2}{\sqrt{2^4}}(\ket{0}+\ket{3}+\ket{5}+\ket{6})$が得られます。
これは、すべてが1ビットの偶数である数値の重ね合わせです。

プレーヤーのアクションによって状態が状態$\ket{b_0^N}$に進化し、各プレーヤーがアダマール行列を自分のキュービットに適用すると、結果はすべての奇数ビット状態(奇数の1ビットの状態を意味する)の重ね合わせになります。

\begin{equation}
\label{204}
(\mathbf{H} \otimes ^N)
\ket{b_2^N}
=
\frac{1}{\sqrt{2^{N-1}}}  \sum_{odd \ bit \ y}^{2^{N-1}} \ket{y} .
\end{equation}

それで、ここに、ゲーム$\Gamma_N$での彼または彼女のキュービットに関して各プレーヤーがとるステップがあります:

プレーヤーのステップ$2a$:プレーヤーがキュービット$x_i = 1$を受け取った場合、プレーヤーは絡み合ったベル状態$\ket{b_0^N}$で自分のキュービットに$ u_{\frac{\pi}{2}}$を適用します。
それ以外の場合、プレーヤーは何もしません。 結果:ビット$ \sum_i x_i$の合計が偶数であるため、偶数のプレーヤーがこのステップを実行します。
$\sum_i x_i$が$4$で割り切れる場合、ベル状態$\ket{B_0^N}$は変更されません。
しかし、$ \sum_i x_i \mod 4$の場合、$ \ket{b_0^n} \to \ket{b_2^N}$です。
プレーヤーステップ$2b$:各プレーヤーはアダマール行列$H$を自分のキュービットに適用します。
結果:エンタングル状態がまだステップ$2a$の状態$\ket{b_0^n}$にある場合、この現在のステップはエンタングル状態をすべての偶数ビット状態の重ね合わせに変換します。
しかし、エンタングル状態が$\ket{b_2^n}$に変換されている場合、このステップはエンタングル状態をすべての奇数ビット状態の重ね合わせに変換します。

プレーヤーのステップ$2c$:各プレーヤーは、計算ベース($\ket{0}$対$\ket{1}$)で自分のキュービットを測定して、$y_i$を生成します。


$\sum_i x_i$が4で割り切れる場合、絡み合ったキュービットは偶数ビット状態の重ね合わせであるため、測定の下で偶数1ビットの数に投影されます。
$\sum_i y_i \mod 2 = 0 $であるため、プレーヤーが勝ちます。$ \sum_i x_i =2 \mod 4$の場合、絡み合ったキュービットは奇数ビット状態の重ね合わせになります。
そのため、測定の下で1ビットの奇数の数値に投影されます。 
$\sum_i y_i mod 2 = 1$であるため、プレイヤーは再び勝ちます。

プレイヤー間のコミュニケーションがなかったとしても、プレイヤーはお互いが何をしているのかを知っているかのように振る舞うことで、疑似テレパシーを示しました。
これは、量子見えざる手として機能する共有エンタングル状態$b_0^N$によって可能になりました。

この疑似テレパシーゲームは、従来の$N$人ゲーム理論の観点から次のように特徴付けることができます。 プレイヤーは自分で値を確保できないため、$1$人の連立$ \{ i \}$の値は$0$: $\nu \{ i \} = 0$です。すべてのプレイヤーの連立の値は$1$: $\nu(N)= 1$です。 ゲームは$(0,1)$-正規化にあると言われます。 $S$をプレーヤー$N$のセットのサブセットとします。すべての$S \subset N$について、$\nu (S) = 0$または$\nu (S) = 1$の場合、ゲームは単純であると言われます。
したがって、疑似テレパシーゲームも単純です。 実際、すべての$S$に対して$\nu (S)= 0$であり、$S = N$を除きます。
最後に、$\nu (S) + \nu (N - S) = \nu (N)$の場合、ゲームは一定の合計であると言われます。
疑似テレパシーゲームは、$S \ne N $に対して$ \nu (S) + \nu (N-S) =0$であるため、一定の合計ではありませんが、$\nu (N) =1$です。

このゲームの帰属のセットは、確率ベクトルのセット$P = \{ p1, p2, \cdots , p_N \}$です。
これは、すべての$i \in N$について、$ \sum_{i \in N} p_i = \nu(N) =1 $であるという要件と、$p_i \ge v( \{ i \})  = 0 $ であるという要件を満たします。
$S \subset N$の場合、これらの割り当てベクトルはいずれも別の割り当てベクトルによって支配されません。したがって、このゲームのコアは、確率ベクトルPの凸集合です。

