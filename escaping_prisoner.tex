\section{Escaping prisoner’s dilemma in a quantum game 量子ゲームの囚人のジレンマ}

これで、量子ゲームを暫定的に定義するのに十分な背景ができました。 量子ゲームΓは、
$\Gamma = \Gamma ( \mathbf{H}, \Lambda,\{ s_i \}_j, \{  \pi_i \}_j ).$
の要素を持つ2人以上のプレーヤー間の相互作用です。$\mathbf{H}$はヒルベルト空間、$\Lambda$はゲームの初期状態、$ \{s_i\}_ j$はプレーヤー$j$の動きのセット、{πi} jはプレーヤーjへのペイオフのセットです。
ゲームの目的は、プレーヤー$j$へのペイオフを最大化する戦略を内生的に決定することです。
  そうする過程で、ゲームとの均衡、およびプレーヤー$j$に対するゲームの値$\bar{\pi}_j$を決定する場合としない場合があります。

最終的な詳細はのちほど述べますが、この時点で、囚人のジレンマの量子バージョンを紹介したいと思います。
  囚人のジレンマの量子バージョン(参考文献20)では、アリスとボブはそれぞれキュービットを持っており、自分のキュービットを操作することができます。
各キュービットは基底ベクトル$\ket{C}$と$\ket{D}$を持つ$\mathbf{H}_2$にあり、ゲームは基底ベクトル$\ket{CC}, \ket{CD}, \ket{DC}, \ket{DD}$を持つ$\mathbf{H}_2 \otimes \mathbf{H}_2 $にあります。
  アリスのキュービットは各ペアの左端のキュービットであり、ボブのキュービットは右端です。 ゲームは単純な量子ネットワークです。
 
 ゲームの初期状態$\Lambda$は
 
\begin{equation}
\label{119}
\Lambda = U \ket{CC},
\end{equation}
 
 ここで、Uは、アリスとボブの両方に知られている、両方のキュービットを操作するユニタリ演算子です。
アリスとボブは戦略的な動きとして$s_A, s_B$,

\begin{equation}
\label{120}
s_A = U_A 
\end{equation}

\begin{equation}
\label{121}
s_B = U_B 
\end{equation}

ここで、UAとUBは、それぞれのプレーヤーのキュービットでのみ動作するユニタリ行列です。 アリスとボブが動き出した後のゲームの状態は

\begin{equation}
\label{122}
( U_A \otimes U_B ) U \ket{CC}.
\end{equation}

アリスとボブは、最終測定のためにキュービットを転送します。 ユニタリ演算子Uの逆数が適用され、ゲームが次の状態になります。

\begin{equation}
\label{123}
U^{\dagger} ( U_A \otimes U_B ) U \ket{CC}.
\end{equation}

次に、測定が行われ、$\mathbf{H}_2 \otimes \mathbf{H}_2$の4つの基底ベクトルの1つが生成されます。 アリスとボブに関連するペイオフ値は、前に表6に示したものです。

アリスとボブがそれぞれのユニタリ行列$U_A, U_B$を選択することにより、この量子ゲームで囚人のジレンマをどのように回避するかは、エンタングルメント関連の戦略を実行することに依存します。 したがって、次のセクションでエンタングルメントを検討するまで、量子囚人のジレンマゲームのさらなる議論を延期します。 ただし、純粋な量子戦略は、プレーヤーのキュービットに作用するユニタリ作用素であるということを強調したかったのです。
