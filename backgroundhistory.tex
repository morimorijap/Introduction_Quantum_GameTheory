%https://resou.osaka-u.ac.jp/ja/story/2020/specialite_002_2

\section{Some background history: 歴史的な背景}

ゲーム理論は公には1944にフォン・ノイマンとモルゲンシュタインによる「ゲームの理論と経済行動」
(The Theory of Games and Economic Behavior, by John von Neumann and Oscar Morgenstern)にて、はじまったとされている。しかし、それには、ハンガリーの数学者ノイマンにより、早い時期にゲーム理論と量子力学の基礎に同時に興味を見出し先例を止めたところにある。量子ゲームに興味があるので、その開発について簡単に説明していきます。

%https://staff.aist.go.jp/akinaga.hiro/Lecture2.pdf

1900年にマックスプランクは、暗黙の合意されている無限エネルギーを取り除くことを試みました。黒体放射の当時の公式は、電磁気学、放射エネルギーは、離散エネルギー単位または量子、倍数でのみ放出または吸収されることを提案しました。
基本単位$h$ : $h \nu$, $2h \nu$, $3h \nu$...
ここで、$\nu$は放射発振器の周波数、$h$は現在、プランク定数として知られています
1905年、アルバートアインシュタインは光電効果をプランクの量子を説明として使用しました。
それにより、電子を放出する前に、金属は最小周波数の入射光を必要としました。
周波数$\nu$の入射光は、粒子(光電子)を集めたように振る舞いが確認できる。そしてエネルギー $E = h \nu$を伴います。
ニールス・ボーアは、満足できない場合でも、有用なものを開発しました。
軌道が仮定された惑星の電子に囲まれた原子核としての原子のモデルで、
角運動量の離散値のみで、プランクの量子のエネルギーの倍数に対応します。
:$ \frac{h}{2 \pi}, \frac{2h}{2 \pi},\frac{3h}{2 \pi} $ ...
1924年ルイ・ド・ブロイは、物質と波との結びつきについて、絵を明確にするのを助けしました。
閉ループの波を記述しました。
原子核を「周回する」電子などは、ループの周りに均等にフィットする必要がありました。
つまり整数サイクルを持ちます。$1,2,3,..$の整数サイクルです。
したがって、プランクの量子(定数$a$の倍数)に関連付けられていました:$1ah,2ah,3ah,...$ これは古い量子論でした。

新しい量子理論は1925年に、ハイゼンベルクは、時間依存の複素数のセットによって物理的な量を表すことを考えました。
ハイゼンベルグの行列力学は本質的に$N \times N$の入出力行列$H$を含んでいて、物質の状態間遷移を表します。
時間$t$で関心のあるシステムの状態を$\psi$で表す(今のところ$t$をゼロに設定します)と、$\psi$は$N \times 1$のベクトルで、ハイゼンベルグは固有ベクトル-固有値システムを使用していました。

\begin{equation}
\label{1}
H \psi = E \psi
\end{equation}

ここで、Eはスカラーで、量子化されたエネルギーレベルを表します。
N方程式のシステムが非縮退であると仮定し、$E$には$N$個の解があります。たとえば、$E_n, n = 1,2,...,N$です。
$E_N$は固有値、つまりエネルギー準位は、$\psi$の状態空間の$N$固有ベクトル基底に関連付けられています。

翌年、エルヴィン・シュレーディンガーは、同じ現象の電磁的解釈を探して、彼の有名な波動方程式を発表しました

\begin{equation}
\label{2}
i \hbar \frac{\partial \psi}{\partial t} 
= \frac{- \hbar^2}{2m}
\left(    
\frac{\partial^2}{\partial x^2} + \frac{\partial^2}{\partial y^2} + \frac{\partial^2}{\partial z^2}   
\right) \psi + V \psi,
\end{equation}

ここで、$ i = \sqrt{-1} $であり $\hbar$はプランクのエネルギー量$h$を$2 \pi$で割ったもので、
$V$はポテンシャルエネルギーである。

シュレーディンガーは彼のアプローチとハイゼンベルグの行列力学が数学的に同等であることを発見し、喜びました。
この同等性の1つの形式は、$\frac{- \hbar^2}{2m} = H \psi$方程式によって示唆されています。
もし、私たちが、$(\ref{2})$のシュレーディンガー方程式の中に、$\psi = A exp^{(-i \frac{E}{\hbar}t)}$、そして、
$ H = \frac{- \hbar^2}{2m}
\left(    
\frac{\partial^2}{\partial x^2} + \frac{\partial^2}{\partial y^2} + \frac{\partial^2}{\partial z^2}   
\right) \psi + V $とした場合、$ E \psi = H \psi $を得ることができ、すなわち $(\ref{1})$のハイゼンベルグの式である。

数年後、ジョン・フォン・ノイマンは、ハイゼンベルクに刺激され、量子力学への関心ができ、
「量子力学はヒルベルト空間におけるエルミート演算子の微積分として形式化できること、そしてハイゼンベルグとシュロディンガーの理論はこの微積分の単なる特定の表現であることを示した。」 
エルミート行列(参考文献35のP22)に示すもので、それ自身が複素共役転置である。
例えば、行列 $ \sigma_y = \left(  
\begin{array}{ccc}
 0 & -i \\
 i & 0 
\end{array}
\right) $ を考える。これの転置行列は
 $ \sigma_y^{\mathrm{T}} = \left(  
\begin{array}{ccc}
 0 & i \\
 -i & 0 
\end{array}
\right) $ 
である。
そして、複素共役とると、虚数は$i \rightarrow -i, -i \rightarrow i $となる。そしてまた$\sigma_y$を得ることができ、$\sigma_y$はエルミート行列である。エルミート行列は、ヒルベルト空間のベクトルの演算子と見なすことができる。
ヒルベルト空間は、複素数$\mathbb{C}$で定義された単なるベクトル空間であり、定義されたノルム、長さ、または内積があることです。
ベクトル$\psi$の場合、ノルムは$ \parallel \psi \parallel = \sqrt{\psi^{\dagger} \psi} $のようになります。$\psi^{\dagger}$は複素共役の転置です。
ヒルベルト空間は無限次元である可能性がありますが、このエッセイでは有限次元空間のみを考慮します。

その間にゲーム理論は生まれました。「ゲーム」という名前は、1921年にフランスの数学者エミールボレルによって導入されました。ポーカーでのブラフに夢中になり、「la théorie du jeu(フランス語でゲーム理論)」を始めた人です。
カールメンガーがウィーン学団のために書かれた1928年の論文49で、フォンノイマンは2人のゼロサムゲームを定義し、完全に解決しました。彼は、協力(連携)の可能性のためにもっと複雑だったN人のゲームについて推測しました。それは3人以上で協力することで恩恵を受ける人もいます。
その後、1932年にプリンストン経済クラブに提出された有名な論文で、同じ年に量子力学の基礎に関する彼の本が出版されました。フォンノイマンは、線形計画法に必要な道具となるもの一式と、後に、モーゲンシュテルンとの共著になるゲーム理論の書籍の基礎を説明しました。(この文章(参考文献51)は1937まで出版されませんでした。)

多くの結果の中心は、線形計画問題とその双対問題(参考文献24)でした。
線形計画問題は次のような問題です。
$m \times n$行列$A$、$n \times 1$のベクトル$b$、および$m \times 1$ベクトル$c$が与えられた場合、次のような非負の$m \times 1$ベクトル$x$を見つる問題です。

\begin{equation}
\label{3}
x^{\mathrm{T}}c \text{ is a maximum}
\end{equation}

\begin{eqnarray}
\label{4}
\text{subject to} \nonumber \\
x^{\mathrm{T}} A \leq b^{\mathrm{T}} .
\end{eqnarray}

双対問題は、次のような非負の$n \times 1$ベクトル$y$を見つけることです。

\begin{equation}
\label{5}
y^{\mathrm{T}} b  \text{ is a minimum}
\end{equation}

\begin{eqnarray}
\label{6}
\text{subject to} \nonumber \\
Ay \leq c.
\end{eqnarray}

フォンノイマン-モルゲンシュテルンから欠落している唯一の主要なゲーム理論の結果
(そして確かに量子ゲーム理論の文献から欠落しているもの)は、コア(参考文献40の8章)の概念です。
コアは$N$人のゲーム理論で発生します。
$N$人のゲーム理論では、プレーヤーの利益は必ずしも反対ではありません、
一部のプレイヤーは、他のプレイヤーと連合を形成することにより、(期待される)ペイオフを改善する場合があります。
プレーヤーのサブセットごとに最大値を決定できます。これにより、ゲームの特性関数が生まれます。
$S$を$N$のサブセットのセットのメンバーとします。
特性関数$\upsilon (S)$は、プレーヤーのサブセット(つまり連合)のセットから実数$\mathbb{R}$のセットの(予想される)ペイオフ値へのマッピングです。

\begin{equation}
\label{7}
\upsilon (S) : S \rightarrow R.
\end{equation}

値$\upsilon (S)$は、連合$S$と残りのすべてのプレーヤーの連合$N - S$との間の2人のゲームで$S$が取得できる最大値として決定されます。
データの補完は、$N$の各プレーヤー$i$に割り当てられた一連の数字(割り当てまたはペイオフ)${\pi i}$です。
コア$C_x$は、次のような代入$C_x = {{\pi i} x}$のセットで、

\begin{equation}
\label{8}
\upsilon (S)  \leq  \sum_{i \in S} \pi_i \ for \ every \ subset\ S \ in \ N, \ and \ \sum_{i \in N} \pi_i = \upsilon (N).
\end{equation}

となります。

コア(空の場合もある)は経済均衡にとって重要です。
コアは、連合の価値を、連合の各メンバーに個別に帰属するペイオフの合計以下に制限します。

DebreuとScarf (参考文献12)は、複製された市場ゲームでは、コアが一連の代入に縮小することを示しました。これは、限界として出現する価格システムの観点から解釈できます。

一方、量子力学では、決定論の反動力が働いていた。
1935年の論文(参考文献18)で、アインシュタイン-ポドルスキー-ローゼン(EPR)は、異なる方向に進む粒子の絡み合った(エンタングルな)ペアを検討することにより、量子力学の不完全性を証明しようとしました。粒子は光年によって分離される可能性があります。
それにもかかわらず、一方の粒子の測定は、もう一方の粒子の状態に即座に影響を与えます。これは、量子力学のアインシュタインが言うところの「不気味な遠隔作用」の例です。
% https://eow.alc.co.jp/search?q=einstein%27s+spooky+action+at+a+distance
(エンタングルメントについては、このエッセイの本文で後で説明しますが、波動関数をテンソル積として記述できない場合、本質的に2つの粒子がエンタングルメントされます。)
この瞬間的な効果は「EPRチャネル」と呼ばれることもありますが、EPRが反対しているのに対し、ボーアはその存在を主張しているため、適切に言えばボーアチャネルと呼ばれるべきです。
ジョン・ベル[1]は、量子力学が不完全であるかどうか、または物理学が非局所的であるかどうかを実験的に区別する一連の不等式を定式化し、いくつかの原因のいくつかの効果の瞬間的な伝播(プロパゲーソン)を可能にしました。
幸いなことに、実験的証拠が決定的に実証されているように、ボーアは正しく、EPRは間違っていました(参考文献25)。ボーアチャネルは現在、量子テレポーテーションの基礎であり、実際、すべての量子コンピューターは、ある意味でボーア効果のデモンストレーションです。

現在のところ、量子ゲーム理論はおそらく量子計算のサブブランチと見なすことができます。
後者の開発に関しては、量子コンピューターの異常な力を最初に予見したのは明らかにリチャード・ファインマン(参考文献22)であり、古典的なコンピューターでの量子進化のシミュレーションは時間の指数関数的な減速を招くと指摘した。子ゲーム理論は、おそらく量子計算のサブブランチと見なすことができます。
 量もう一度、フォンノイマン(参考文献52)(スタンウラム(参考文献68)との)から直接的な関係があり、以下のように述べています。
 「1950年代に、スタニスワフ・マルチン・ウラムとフォンノイマンは、セルオートマトンと呼ばれる計算モデルについて議論し始めました。このモデルでは、多くの自由度を持つシステムに適用される単純な計算規則によって、複雑な動作パターンが生成される可能性があります。彼らの考えの根底にあるのは、連続的な空間と時間に基づく物理学の従来の記述に対する不満でした。」(参考文献34)
 
デービッドドイチェ(参考文献13)は、量子\textit{重ね合わせ}が多くの古典的な計算の並列実行を可能にするかもしれないと示唆しました。確かに、重ね合わせは、もつれの状態どうかに関係なく、量子ゲームを古典的なゲームとは異なるものにする重要な新しい要素であることがわかります。動的ゲームの場合、重ね合わせで十分ですが、静的ゲームでは通常、エンタングルメントも必要です。動的ゲームの場合、重ね合わせで十分ですが、静的ゲームでは通常、エンタングルメントも必要です。(重ね合わせは、同時に2つ以上の状態の線形結合になることができる量子の能力です。)

量子計算への関心の嵐を生み出した「キラーアプリ」は、ピーターショア(参考文献62)が、量子力学アルゴリズムが多項式時間で数値を因数分解できることを示したときに生まれました。これは、従来のコンピューターで利用可能な素因数分解アルゴリズムよりも指数関数的に高速化されました。ショアのアルゴリズムは、主に重ね合わせと量子フーリエ変換の独創的なアプリケーションに依存しています。

別の結果は、ロブ・グローバー(参考文献28)によって示されました。彼は、$O(N)$ステップから$O(\sqrt{N})$ステップまで$N$アイテムデータベース内のアイテムの検索を高速化する量子力学的方法を示しました。 グローバーの結果は、ヒルベルト空間での量子状態(ベクトル)の回転に基づいています。

量子ゲーム理論は、デビッド・マイヤーがマイクロソフトでこのテーマについて講演したときに具体化したようです(これについては(参考文献46)を参照)。このエッセイで検討されている12の量子ゲームのうち、3つはマイヤーによるものです(スピンフリップゲーム、および数ゲーム$I$と$II$です。)

フォンノイマンとモルゲンシュテルンが以下のように述べたように(参考文献53)、「私たちが経済学に適用している概念を解明するために、物理学からいくつかの実例を示しました。
さまざまな理由でそのような類似点を描くことに反対する多くの社会科学者がいますが、その中には、経済理論は社会科学であり、人間の現象の科学であるため、心理学を考慮に入れる必要があるため、物理学に基づいてモデル化することはできないという主張が一般的に見られますが、そのような声明は少なくとも時期尚早です。」
逆に、経済概念と量子力学の概念を混合することに同様に反対する人もいるかもしれませんが、そのような反対は少なくとも時期尚早です。
確かに、人間の脳は間違いなく量子コンピューターです(参考文献65, 66, 55, 14, 15)、精神はそれ以上かもしれませんが、心理学の問題で量子力学を無視することは、経済学ではなく、とても愚かなことです。 
逆に言えば、量子測定問題における人間の精神の役割は、フォン・ノイマンによって最初に明確に描写されて以来、論争の的となっています(参考文献36)。 いずれにせよ、量子ゲームには経済学と量子力学の両方の教訓があるかもしれません。

