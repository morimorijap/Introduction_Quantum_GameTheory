\section{Quantum cloning 量子クローニング}

計量経済学では、いくつかの手順によって、いくつかの未知のパラメータ$a$の推定値$\hat{a}$を生成しようとします。
これは、推定手順によると、パラメーター$a$のクローンを作成する試みと見なすことができます。
完全なクローンを作成することは期待していませんが、不確実性の範囲内にある最良の見積もりのみです。
これにより、量子状態の複製が可能になります。
最適なクローニングデバイス(参考文献69)の目的は、オリジナルにできるだけ近いコピーを作成することです。

最適なクローン作成は、アリスとクローン作成の女王であるクレアの間で行われる量子ゲーム、クローン作成ゲームの観点から定式化できます。
このゲームには、$N$個の入力システムと$M$個の出力システムがあります。 まず、2次元ヒルベルト空間$\mathbf{H}_2$の密度行列$\rho$で記述された純粋な状態を持つアリスから始めます。
彼女は状態準備手順を$N$回実行し、ヒルベルト空間$\mathbf{H}_2 \otimes ^N$で複合システムを生成します。

\begin{equation}
\label{260}
\mathbf{1}_2 \otimes ^N \rho =  \rho \otimes ^N  .
\end{equation}

その後、アリスは$ \rho \otimes ^N$をクレアに発送します。 クレアは、選択したクローンデバイス$T_m $を使用して、$M$個の出力システム$T_m \rho \otimes ^N $を生成します。
次に、アリスは元のシステム$ \rho \otimes ^M $の$M$個のコピーを作成します。 ゲームの結果は

\begin{equation}
\label{261}
T_m \rho \otimes ^N  vs. \   \rho \otimes ^M  .
\end{equation}

$T_m$は密度行列を密度行列にマップするため、線形の完全に正のトレース保存マップに制限されます。

このゲームにペイオフを割り当てる1つの方法は、ノルム基準の違いに基づいてペイオフを行うことです。

\begin{equation}
\label{262}
|| T_m \rho \otimes ^N  -  \rho \otimes ^M  ||.
\end{equation}

もう1つの方法は、$\mathbf{trace}( \rho \otimes ^M  T_m \rho \otimes ^N)$ に基づいて fidelity(忠実度)を使用することです。
クローン作成機が完璧だったら、これは$1$になります。
 fidelity(忠実度)は、入力密度行列ρに依存する可能性があります。
F(T)を次のように定義します

\begin{equation}
\label{263}
F(T) = \inf_{\rho} trace ( \rho \otimes ^M  T_m \rho \otimes ^N) < 1.
\end{equation}

次に、クレアの仕事は$F(T)$を最大化することです。 これにより、クローンゲームが最大の問題になります。
出力クローンの忠実度が入力状態に依存しない場合、クローン作成者は「ユニバーサル」と呼ばれます。
ユニバーサルクローン作成者のクローン作成の最大忠実度は5であり、これは単一進化またはテレポートスキームによって達成できます(参考文献8)。

$1 qubit \to 2 qubit$のユニバーサル量子クローンは、未知の量子状態$\ket{\psi}$を入力として受け取り、$\rho = \eta \ket{\psi} \bra{\psi} + (1 - \eta) \frac{1}{2} \mathbf{1} $の形式の密度行列で記述できる状態で2キュービットを出力として生成する量子機械です。
パラメータ$\eta$は、密度演算子$\ket{\psi} \bra{\psi} $に対応する元のブロッホベクトル$r$の縮小を表します。
たとえば、$\ket{\psi} \bra{\psi} = \frac{1}{2} ( \mathbf{1} + \mathbf{r} \cdot \sigma)$の場合、$ \rho = \frac{1}{2} ( \mathbf{1} + \eta \mathbf{r} \cdot \sigma)$
次に、最適なクローン作成者は、$ \eta <1$を最大化することによって fidelity(忠実度)を最大化することを含みます。

\begin{equation}
\label{264}
\max_{\eta} F = \braket{\psi | \rho | \psi} = \frac{1}{2}(1 + \eta).
\end{equation}

$\eta= \frac{2}{3}$のブロッホ球のベクトル収縮は$\frac{5}{6}$の最大fidelity(忠実度)に対応します。

クローン作成プロセスは次のようになります。 $\ket{B}$は、空白のコピーの初期状態(クローンの宛先)と、プロセスで必要な補助量子ビット(「アンシラ」)を示します。 
複製されるキュービット$\ket{\psi}$は、基底$(\ket{0}, \ket{1})$でエンコードされます。 
次に、ユニバーサル量子クローニングマシン(UQCM)変換$T_{UQCM}$は、基底ベクトルまたは状態に基づいて次の変換を実行します。

\begin{equation}
\label{265}
T_{UQCM} \ket{0} \ket{B} 
\to 
\sqrt{\frac{2}{3}} \ket{0} \ket{0} \ket{A_\bot}
+
\sqrt{\frac{1}{6}} ( \ket{01} + \ket{10} )\ket{A}
\end{equation}
% 常に真となる論理式(例えばa→a)をトートロジーとよび、⊤で表す。また矛盾を表す記号は⊥となる。

\begin{equation}
\label{266}
T_{UQCM} \ket{1} \ket{B} 
\to 
\sqrt{\frac{2}{3}} \ket{1} \ket{1} \ket{A}
+
\sqrt{\frac{1}{6}} ( \ket{01} + \ket{10} )\ket{A_\bot}.
\end{equation}

ここで、$A$と$A_\bot$は、アンシラキュービットの2つの可能な直交最終状態を表します。 これは、入力状態$\ket{\psi}$、出力を意味することに注意してください。

\begin{equation}
\label{267}
T_{UQCM} \ket{\psi} \ket{B} 
\to 
\end{equation}

\begin{equation}
\label{268}
\left( \begin{array}{cc}
\sqrt{\frac{2}{3}} \ket{0} \ket{0} \ket{A_\bot}
+
\sqrt{\frac{1}{6}} ( \ket{01} + \ket{10} )\ket{A},
&
\sqrt{\frac{2}{3}} \ket{1} \ket{1} \ket{A}
+
\sqrt{\frac{1}{6}} ( \ket{01} + \ket{10} )\ket{A_\bot} )
\end{array} \right)
\left( \begin{array}{cc}
a \\
b
\end{array} \right).
\end{equation}

次のステップは、2量子ビットの混合状態を生成する、補助量子ビットをトレースすることです。
次に、個々のキュービットごとに別のトレースが実行され、同じ混合1キュービット状態の2つのコピーが得られます。これは、元の状態と比較した場合のfidelity(忠実度)が$\frac{5}{6}$です。





