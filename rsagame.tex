\section{The RSA game}

RSAは、銀行やその他の場所で広く使用されている暗号化システムです。
整数環$Z_N$について考えてみます。ここで、2つの異なる大きな素数$p$と$q$の$N = pq$です。
暗号化の場合、RSAは$Z_N$の単位のみを許可します(つまり、$Z_N$から$p$または$q$のすべての倍数を削除します)。
$Z_N^*$と呼ばれる残りの整数のセットは、オイラーの$\phi $関数(オイラーのトーティエント関数)$\phi = (p-1)(q-1)=(n + 1)-(p + q)$と呼ばれ、乗算中のアーベル群です。
RSA暗号システムは、比較的小さい奇数の整数$e$を選択し、$d = e^{-1} \mod \phi$を計算します。
次に、$Z_n^*$のメッセージ$M$は$M^e \mod N$として暗号化され、$M^{ed} = M^{\phi + 1} = M \mod N$として復号化されます。
番号$e$と$N$は公開されていますが、復号化キー$d$はメッセージの受信者だけが知っています。

アリスはボブに次のゲームに挑戦します。 彼女は公開鍵$N$と$e$を作成し、メッセージ$M$を暗号化します。3つのコンポーネント$(N,e,M^e)$がボブに送信されます。 ボブが$( \log N^3$のステップ以内にメッセージ$M^e \to M$を復号化できる場合、ボブは1,000ドルを獲得します。 そうでなければ、彼は1,000ドルを失います。

現在、RSAは非常に大きな数$N$を使用しています。
ただし、Shorのアルゴリズムの手順を説明するために、非常に単純な例を使用します。
アリスがボブにトリプレット$(77,11,67)$を送信すると仮定します。
最初に、$77^2 = 5929$、および$2^{12} < 5929 < 214$であることに注意してください。
左側の量子レジスタには14キュービットが必要であり、右側のレジスタには7キュービットが必要です。

ステップ1:ボブはランダムに$m = 39$を選択します。ここで、$2 \le 39 \le 75$です。$gcd(39,77)= 1$なので、ボブはステップ2に進みます。

ステップ2:左側の量子ビットレジスタで、ボブは$0$から$16383 = 2^{14} −1$までのすべての数値の重ね合わせを作成します。

ステップ3:ボブは、重ね合わせの各$x$に関連付ける変換$f_m$、値$39^x \mod 77$を適用します。$39^{30} \mod 77= 1$なので、$x \in S = {30,60,90,120,150, \cdots ,16380}$。 つまり、$m = 39$の周期は$r = 30$です。しかし、ボブはまだこれを知りません。

ステップ4:ボブは、$x$の値を含む左側のレジスターで量子フーリエ変換を実行します。 次に、両方のレジスタを観察し、左側のレジスタの状態で$w = 14,770$、右側のレジスタの$39^z \mod 77$の値で$Z = 53$を取得します。

ボブは、分母が$77$未満の$\frac{14770}{16384}$に最も近い分数を見つけたいと考えています。
この分数は27に非常に近いため、ボブは$r'= 30$、または$r' = 15$を試します。 彼は$39^{15}-1 \mod 77 = 42$、$39^{15} + 1 \mod 77 = 44$、および$gcd(77,2) = 7$、$gcd(77,44) = 11$を取得します。
これらの2つの要素を使用して、ボブは$\phi =(7 − 1)(11 − 1)= 60$を計算します。したがって、復号化キー$d$の場合、$d = e^{−1} \mod 60$が必要です。これにより、$d = 11^{−1} \mod 60 = 11$。
復号化キーは暗号化キーと同じです。 (これは、使用したモジュラス$N = 77$が非常に小さいためです。)ボブは、アリスの暗号化されたメッセージ$(M^e)^d = 67^{11} \mod 77 = 23$を復号化します。ボブはアリスにメッセージ$M = 23$を伝え、1,000ドルを収集します。
