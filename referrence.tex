%% 参考文献もちゃんとDB化した方が良いが労力の関係でやっていません。
\begin{verbatim*}

[1] Bell J.S., ‘On the Einstein Podolsky Paradox’, Physics, 1(3), 1964, 195-200.
[2] Bennett Charles H., Gilles Brassard, Claude Cre ́peau, Richard Jozsa, Asher Peres, William
K. Wootters, ‘Teleporting an unknown quantum state via dual classical and EPR channels’,
http://www.enricozimuel.net/documenti/BBC+93.ps .
[3] Bernstein E. and U. Vazirani, ‘Quantum complexity theory’, in Proceedings of the 25th Annual ACM
Symposium on the Theory of Computing, San Diego, Calif., 16-18 May 1993, New York:ACM, 1993,
11-20, http://www.cs.berkeley.edu/∼vazirani/pubs/bv.ps
[4] Brams Steven J., Superior Beings: If they exist, how would we know? Game theoretic implications of
omniscence, omnipotence, immortality, and incomprehensibility, New York:Springer-Verlag, 1983.
[5] Brassard Gilles, Anne Broadbent, Alain Tapp, ‘Recasting Mermin’s multi-player game into the frame-
work of pseudo-telepathy’, arXiv: quant-ph/0408052 v1 6 Aug 2004.
[6] Braunstein Samual L., ‘Quantum Computation’, http://www-users.cs.york.ac.uk/∼schmuel/comp
/comp best.pdf .
[7] Braunstein Samuel L. and H. J. Kimble, ‘Teleportation of continuous quantum variables’, Physical
Review Letters 80, 4, 26 January 1998, http://www-users.cs.york.ac.uk/∼schmuel/papers/bk98.pdf
[8] Bruß Dagmar, David P. DiVincenzo, Artur Kert, Christopher A. Fuchs, Chiara Macchiavello, John A. Smolin, ‘Optimal universal and state-dependent quantum cloning’, arXiv: quant-ph/9705038 v3 6
Dec 1997.
[9] Chefles Anthony, ‘Quantum state discrimination’, arXiv: quant-ph/0010114 v1 31 Oct 2000.
65
  [10] Cheon Taksu and Izumi Tsutsui, ‘Classical and quantum contents of solvable game theory on Hilbert space,’ arXiv; quant-ph/0503233 v1 31 Mar 2005
[11] Cleve Richard, Daniel Gottesman, Hoi-Kwong Lo, ‘How to share a quantum secret’, December 1998,
http://www.hpl.hp.com/techreports/98/HPL-98-205.pdf
[12] Debreu G. and H. E. Scarf, ‘A limit theorem on the core of an economy’, International Economic
Review, 4, 1963, 235-246.
[13] Deutsch D., ‘Quantum Theory, the Church-Turing principle and the universal quantum computer’,
Proc. Roy. Lond. A400, 1985, 97-117.
[14] Deutsch, D., ‘Quantum computational networks,’ Proceedings of the Royal Society of London, A425,
1989, 73-90.
[15] Deutsch,D.,‘ItfromQubit’,Sept.2002,http://www.qubit.org/people/david/Articles/ItFromQubit.pdf
[16] Deutsch D. and R. Jozsa, ‘Rapid solution of problems by quantum computation,’ Proceedings Royal
Society London, A400, 1992, 73-90.
[17] Du Jianfeng, Xiaodong Xu, Hui Li, Mingjun Shi, Xianyi Zhou, Rongdian Han, ‘Quantum strategy
without entanglement’, arXiv: quant-ph/0011078 v1 19 Nov 2000.
[18] Einstein A., B. Podolsky, N. Rosen, ‘Can quantum mechanical description of physical reality be
considered complete?’, Phys. Rev. 47, 1935, 777-780.
[19] Eisert Jens and Martin Wilkens, ‘Quantum Games,’ arXiv:quant-ph/0004076 v1 19 Apr 2000.
[20] Eisert Jens, Martin Wilkens, and Maciej Lewenstein, ‘Quantum games and quantum strategies’,
arXiv: quant-ph/9806088 v3 29 Sept 1999.
[21] Ekert Artur, Patrick Hayden and Hitoshi Inamori, Basic concepts in quantum computation, arXiv:
quant-ph/0011013 v1 2 Nov 2000,
[22] Feynman Richard P., ‘Simulating Physics with Computers,’ International Journal of Theoretical
Physcis, 21, 1982, 467.
[23] Fishburn Peter C., ‘Expected utility theories: a review note’, in R. Henn and O. Moeschlin, eds.,
Mathematical Economics and Game Theory: Essays in honor of Oskar Morgenstern, Lecture Notes
in Economics and Mathematical Systems, 141, Berlin:Springer-Verlag, 1977.
[24] Gale David, The Theory of Linear Economic Models, New York: McGraw-Hill, 1960.
[25] Gisin Nicolas, ‘How come the correlations?’ http://arxiv.org/ftp/quant-ph/papers/0503/0503007.pdf
[26] Gogonea V. and K. M. Merz, ‘Fully quantum mechanical description of proteins in solution combin-
ing linear scaling quantum mechanical methodologies with the Poisson-Boltzmann equation’, J. Phys. Chem. A, 103 (1999) 51715188.
[27] Gottesman Daniel, ‘The Heisenberg representation of quantum computers’, arXiv: quant-ph/9807006 v1 1 July 1998.
[28] GroverLovK.,‘Afastquantummechanicalalgorithmfordatabasesearch’,arXiv:quant-ph/9605043.
[29] Hardy G. H. and E. M. Wright, An Introduction to the Theory of Numbers, Fifth edition, Ox-
ford:Clarendon Press 1979.
[30] Herbert,N.‘FLASH–asuperluminalcommunicatorbaseduponanewtypeofquantummeasurement”,
Found. Phys. 12, 1982, 1171.
[31] Hillary Mark, Vladimir Buzek, and Andre Berthiaume, ‘Quantum secret sharing’, Physical Review A,
vol 59, no 3, March 1999, 1829-1834, http://www.quniverse.sk/buzek/mypapers/99pra1829.pdf
[32] Hunziker Markus and David A. Meyer, ‘Quantum algorithms for highly structured search problems,’
http://www3.baylor.edu/∼Markus Hunziker/HunzikerMeyer2002.pdf .
[33] Iqbal A. and A.H. Toor, ‘Evolutionary stable strategies in quantum games’, arXiv: quant-ph/0007100
v3 11 Dec 2000.
[34] Jaroszkiewicz George and Jason Ridgway-Taylor, ‘Quantum Computational Representation of the
Bosonic Oscillator’, arXiv:quant-ph/0502166 v1 25 Feb 2005
[35] Jammer Max, The Philosophy of Quantum Mechanics, New York: Wiley, 1974.
[36] Johnson Joseph F., ‘The problem of quantum measurement’, arXiv quant-ph/0502124 v1 21 Feb 2005.
[37] Lambertini, Luca, ‘Quantum mechaics and mathematical economics are isomorphic,’ 29 Feb 2000,
http://www.dse.unibo.it/wp/370.pdf
[38] Lee Chiu Fan and Neil F. Johnston, ‘Game theoretic discussion of quantum state estimation and
cloning’, arXiv: quant-ph/0207139 v2 29 Nov 2002.
[39] Lomonaco, Jr. Samuel J., ‘A lecture on Grover’s quantum search algorithm’, arXiv:quant-ph/0010040
v2 18 Oct 2000.
[40] Luce R. Duncan and Howard Raiffa, Games and Decisions, New York: Wiley, 1957.
[41] Marinatto Luca and Tullio Weber, ‘A quantum approach to static games of complete information’,
arXiv: quant-ph/0004081 v2 27 June 2000 .
[42] Mayer, Gottfried J., Editor’s Note to Complexity Digest, 27, 2 July 2001.
[43] Maynard Smith J. and G.R. Price, ‘The logic of animal conflict’, Nature, 246, 1973, 15-18.
[44] Maynard Smith J., Evolution and the Theory of Games, Cambridge: Cambridge University Press,
1982.
[45] Meglicki, Zdzislaw, ‘Introduction to quantum computing’, February 5, 2002,http://beige.ucs.indiana.edu/M743/M743.pdf .
[46] Meyer David A., ‘Quantum Games and Quantum Algorithms’, arXiv:quant-ph/0004092 v2, 3 May
2000.
[47] Milman P. H. Ollivier, and J. M. Raimond, ‘Universal quantum cloning in cavity QED’,
http://www.imperial.ac.uk/physics/qgates/papers/ENS QG04.pdf, 23 Jan 2003.
[48] Nawaz Ahmad and A. H. Toor, ‘Dilemma and Quantum Battle of the Sexes’,
arXchiv:quant-ph/0110096 v3, 26 Mar 2004.
[49] Neumann John von, ‘Zur Theorie der Gesellschaftspiele’ Mathematische Annalen, 1928. 100:295-
320.
[50] NeumannJohn von, Mathematische Grundlagen der Quantenmechanik, Berlin: Springer-Verlag,
1932.
[51] Neumann John von, ‘A Model of General Economic Equilibrium’ (‘U ̈ber ein o ̈konomisches Gle-
ichungssystem und eine Verallgemeinerung des Brouwerschen Fixpunktsatzes’) in K. Menger, ed.,
Ergebnisse eines mathematischen Kolloquiums, 1935-36, 1937.
[52] Neumann, John von, ‘Probabilistic logics and the synthesis of reliable organisms from unreliable
components’, Automata Studies, Princeton University Press, 1956, 329-378.
[53] Neumann John von and Oscar Morgenstern, The Theory of Games and Economic Behavior, New
York: Wiley, 1944.
[54] Ore Oystein, Number Theory and Its History, New York: Dover (reprint of New York: McGraw-Hill,
1948), 1988.
[55] Penrose, Roger, The Emperor’s New Mind, Oxford: Oxford University Press, 1989.
[56] Peres Asher, ‘How the no-cloning theorem got its name,’ arXiv: quantum-ph/0205076 v1 14 May
2002.
[57] Piotrowski Edward W. and Jan Sladkowski, ‘An invitation to quantum game theory’, arXiv:
quant-ph/0211191 v1 28 Nov 2002.
[58] Piotrowski Edward W. and Jan Sladkowski, ‘Quantum solution to the Newcomb’s paradox’, arXiv:
quant-ph/0202074 v1 13 Feb 2002.
[59] Piotrowski Edward W. and Jan Sladkowski, ‘Trading by quantum rules–quantum anthropic principle’,
http://alpha.uwb.edu.pl/ep/RePEc/sla/eakjkl/9.pdf .
[60] Pirandola Stefano, ‘A quantum teleportation game’, arXiv: quant-ph/0407248 v3 17 Nov 2004.
[61] Preskill John, ‘Lecture notes for Physics 229: quantum information and computation’, California In-stitute of Technology, September 1998, http://www.theory.caltech.edu/people/preskill/ph229/#lecture
[62] Shor P. W., ‘Algorithms for quantum computation: discrete logarithms and factor- ing’, in Proc. 35th Annual Symposium on the Foundations of Computer Science, edited by S. Goldwasser, Los Alamitos, Calif.:IEEE Computer Society Press, 1994, 124-134,
http://www.ennui.net∼quantum/papers/9508027.pdf .
[63] Srednick Mark, ‘Subjective and objective probabilities in quantum mechanics,’ arXiv:
quant-ph/0501009 v2 14 Jan 2005.
[64] Stanford Encyclopedia of Philsophy, ‘Evolutionary game theory’,
http://plato.stanford.edu/entries/game-evolutionary/ .
[65] Stapp Henry, ‘Why classical mechanics cannot naturally accomodate consciousness, but quantum
meachanics can,’ http://psyche.cs.monash.edu.au/v2/psyche-2-05-stapp.html .
[66] Stapp Henry, The Mindful Universe, http://www-physics.lbl.gov∼stapp/MUA.pdf
[67] Turner P.E. and L. Chao, ‘Prisoner’s dilemma in an RNA virus,’ Nature, 398(6726), April 1, 1999,
441-3.
[68] Ulam, S.M., Adventures of a Mathematician, New York:Charles Scribner’s Sons, 1976.
[69] Werner R. F., ‘Optimal cloning of pure states’, arXiv: quant-ph/9804001 v1 1 April 1998.
[70] Zalka Chris, ‘Grover’s quantum searching algorithm is optimal’, arXiv:quant-ph/9711070 v2, 2 Dec 1999.

\end{verbatim*}
