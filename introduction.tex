この文章は、J. Orlin Grabbe著「An Introduction to Quantum Game Theory」arXiv:quant-ph/0506219v1 ,27/Jun/2005の日本語訳です。残念なことにJOGは2008年3月亡くなられているとのことです。ご本人に連絡が取れないので著作権的に問題がありましたらご遺族からでもご連絡いただければ幸いです。

\begin{abstract}

このエッセイは量子ゲーム理論とそれに関連する経済学分野を量子力学の近い関係を紹介するものです。この文章を読むには初歩の代数幾何学の知識が必要です。量子力学のノート、結果を紹介するのに必要だからです。そして、人間の行動の問題をゲーム理論として量子力学の数式を用いて記述することが可能だからです。

キーワード;量子ゲーム理論、量子コンピュータ、経済物理学
\end{abstract}

量子ゲーム理論は量子的なコンピュータ(量子コンピュータ)を作るのに、また古典的な経済学でのゲーム理論、そして量子力学においてとても重要な役割を担います。しかしながら、量子力学及び量子コンピュータの知識は、ほとんどの経済学者とコニュニケーションをするのには大きな壁になります。他の見方をすると、量子ゲーム理論は、古典的なゲーム理論でわかっていることとは異なる認識を示すこともあります。このエッセイは、そのギャップの橋渡しをして、ゲーム理論を勉強している経済学者が自分で、量子ゲーム理論を学べるようにする試みです。
このエッセイはベクトルなどの代数幾何学の知識、量子力学、量子コンピューティングの知識が必要になるでしょう。キーコンセプトはグローバーの探索アルゴリズム、ショアの因数分解アルゴリズム、そして、量子テレポーテーション、量子同士がつながっているような見せかけのテレパシーなど、量子もつれ(量子エンタングルメント)に基づいて詳細を説明し、12の量子ゲーム理論により、古典的なゲーム理論と量子ゲーム理論の違いを解き明かします。これらの途中で、量子力学の中に、たくさんの古典的な問題の中にゲーム理論に基づいた定式化が出てくることになります。

