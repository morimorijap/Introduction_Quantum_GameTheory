\section{Quantum secret sharing 量子秘密共有}

IRAには、メンバー間で保存したい秘密情報がいくつかありますが、一部はMI5の情報提供者であり、他の人は逮捕され、尋問の下で知っていることを明らかにする可能性があることを恐れています。
したがって、彼らは彼ら自身の間に秘密を埋め込むための安全な方法を必要としています。
$(k,n)$しきい値スキーム(参考文献11)は、$ k \le n$ のメンバーはシークレットを再構築できますが、$k-1$のメンバーはシークレットに関する情報をまったく見つけることができないスキームです。

ただし、最初に、秘密の量子状態を発見するために2つのパーティが協力しなければならない簡単な例を考えてみましょう(参考文献31)。 アリス、ボブ、ジェラルドは次のもつれ状態を共有しています(左のキュービットはアリス、右のキュービットはジェラルドです)。

\begin{equation}
\label{205}
\ket{\psi} = \frac{1}{\sqrt{2}}(\ket{000}-\ket{111}).
\end{equation}

最初に、これを別の基準で書き直すことができることに注意してください。 させて

\begin{equation}
\label{206}
\ket{x^{+}} = \frac{1}{\sqrt{2}}(\ket{0}+\ket{1})
\end{equation}

\begin{equation}
\label{207}
\ket{x^{-}} = \frac{1}{\sqrt{2}}(\ket{0}-\ket{1}).
\end{equation}

これは相互関係を意味します

\begin{equation}
\label{208}
\ket{0} = \frac{1}{\sqrt{2}}(\ket{x^+}+\ket{x^-})
\end{equation}

\begin{equation}
\label{209}
\ket{1} = \frac{1}{\sqrt{2}}(\ket{x^+}-\ket{x^-}).
\end{equation}

したがって、新しい基準に関する元の状態は次のようになります。
\begin{equation}
\label{210}
\ket{\psi} = \frac{1}{2\sqrt{2}}[(\ket{x^+ x^+}+\ket{x^- x^-})(\ket{0}+\ket{1})+  (\ket{x^+ x^-}+\ket{x^- x^+})(\ket{0}-\ket{1})].
\end{equation}

アリスは、ボブとジェラルドが秘密を学ぶために協力しなければならないような方法で、秘密のキュービット
$\ket{\phi_{secret}} a\ket{0}  -b\ket{1}$をボブとジェラルドに送りたいと思っています。
彼女は基本的にテレポーテーションプロトコルを介してこれを行いますが、手順の一部として$(\ket{x^+}, \ket{x^-})$の定義も必要になります。 アリスは秘密量子ビット$\ket{\phi_{secret}} $と共有状態$\ket{\psi}$を組み合わせて全体的な状態を形成します

\begin{equation}
\label{211}
\ket{\phi_{secret}} \otimes \ket{\psi} = \frac{1}{\sqrt{2}}(a \ket{0000} + b\ket{1000} + a\ket{0111}+ b\ket{1111}).
\end{equation}

アリスはこれをベル基底の観点から書き直しました。 ベル状態の乗数は次のとおりです。

\begin{equation}
\label{212}
\braket{b_0 | (\ket{\phi_{secret}} \otimes \ket{\psi} )}
= 
\frac{a}{\sqrt{2}} \ket{00} + \frac{b}{\sqrt{2}} \ket{11}
\end{equation}

\begin{equation}
\label{213}
\braket{b_1 | (\ket{\phi_{secret}} \otimes \ket{\psi} )}
= 
\frac{a}{\sqrt{2}} \ket{11} + \frac{b}{\sqrt{2}} \ket{00}
\end{equation}

\begin{equation}
\label{214}
\braket{b_2 | (\ket{\phi_{secret}} \otimes \ket{\psi} )}
= 
\frac{a}{\sqrt{2}} \ket{00} - \frac{b}{\sqrt{2}} \ket{11}
\end{equation}

\begin{equation}
\label{215}
\braket{b_3 | (\ket{\phi_{secret}} \otimes \ket{\psi} )}
= 
\frac{a}{\sqrt{2}} \ket{11} - \frac{b}{\sqrt{2}} \ket{00}.
\end{equation}

アリスはベル基底で2つのキュービットを測定し、その結果をジェラルドに送信し、ボブに$(\ket{x^+}, \ket{x^-})$基底でキュービットを測定するように指示します。 アリスのベルの測定後、ボブとジェラルドのキュービットは次のいずれかの状態になります。

\begin{equation}
\label{216}
\ket{b_0}
\to
a \ket{00} + b \ket{00}
\end{equation}

\begin{equation}
\label{217}
\ket{b_1}
\to
a \ket{11} + b \ket{00}
\end{equation}

\begin{equation}
\label{218}
\ket{b_2}
\to
a \ket{00} - b \ket{11}
\end{equation}

\begin{equation}
\label{219}
\ket{b_3}
\to
a \ket{11} - b \ket{00}.
\end{equation}

ボブが測定時に$\ket{x^+}$を取得した場合、ジェラルドのキュービットは次のようになります。


\begin{equation}
\label{220}
a\ket{00} + b\ket{11}
\to
a \ket{0} + b \ket{1}
\end{equation}

\begin{equation}
\label{221}
a\ket{11} + b\ket{00}
\to
a \ket{1} - b \ket{0}
\end{equation}

\begin{equation}
\label{222}
a\ket{00} - b\ket{11}
\to
a \ket{0} - b \ket{1}.
\end{equation}

\begin{equation}
\label{223}
a\ket{11} - b\ket{00}
\to
a \ket{1} - b \ket{0}
\end{equation}

一方、ボブが$(\ket{x^-}$を取得すると、ジェラルドのキュービットは次のようになります。


\begin{equation}
\label{224}
a\ket{00} + b\ket{11}
\to
a \ket{0} - b \ket{1}.
\end{equation}

\begin{equation}
\label{225}
a\ket{11} + b\ket{00}
\to
a \ket{1} + b \ket{0}
\end{equation}

\begin{equation}
\label{226}
a\ket{00} - b\ket{11}
\to
a \ket{0} + b \ket{1}
\end{equation}

\begin{equation}
\label{227}
a\ket{11} - b\ket{00}
\to
a \ket{1} - b \ket{0}.
\end{equation}


アリスのキュービットを再構築するには、ジェラルドがボブが取得した測定値を知る必要があります。これにより、ジェラルドは適切なパウリスピン行列を最終的なキュービット状態に適用できます。
したがって、ジェラルドとボブは一緒にアリスのキュービットを再構築できますが、どちらも単独で再構築することはできません。 ジェラルドの最終状態に適用される適切なパウリスピン行列は次のとおりです。

\begin{table}[htb]
\caption{ジェラルドの最終的なキュービット状態に適用されるパウリスピン行列}
\centering
\begin{tabular}{|r|r|r|} \hline
$Bell \setminus Bob$  & $(\ket{x^+}$ & $(\ket{x^-}$ \\ \hline
$\ket{b_0}$ & $ \mathbf{1}$ & $\sigma_z$ \\
$\ket{b_1}$ & $ \sigma_x$ & $\sigma_x \sigma_z$ \\
$\ket{b_2}$ & $\sigma_z$  & $\mathbf{1}$ \\
$\ket{b_3}$ & $\sigma_z \sigma_x$  & $- \sigma_x$ \\ \hline
\end{tabular}
\end{table} 

% 表XIV:ジェラルドの最終的なキュービット状態に適用されるパウリスピン行列 \\


量子秘密共有とテレポーテーションの密接な関係を確認したので、少なくとも1つの例では、$(k,n)$しきい値の概念に戻り、$(2,3)$しきい値スキームの例を考えてみましょうこのスキームは、任意の2つが元の状態を再構築できるように、状態を3つのパーティに分割することによって機能します。
まず、キュービット(qubit)ではなく、キュートリット(qutrit)である未知の秘密の状態から始めます。
キュートリット(qutrit)は、$(\ket{0}, \ket{1},\ket{2} )$にまたがる3次元ヒルベルト空間で値をとることができる3進の「trit」です。
キュービットにもう1つのディメンションを追加しただけです。 この例では、テンソル積は3の累乗で拡張するため、3つのキュートリットが次元27のヒルベルト空間を占めることに注意してください。
$\mathbf{H}_{27} = \mathbf{H}_{3} \otimes \mathbf{H}_{3} \otimes \mathbf{H}_{3} $

秘密の状態$\ket{\phi_{secret}} = \alpha \ket{0} + \beta \ket{1} + \gamma \ket{2}.$があります。 この1キュートリット状態を混合3キュートリット状態にマッピングするエンコーディング変換があります。


\begin{equation}
\label{228}
\ket{\phi_{secret}} 
\to
\alpha(\ket{000} + \ket{111} + \ket{222})
+ \beta (\ket{012} + \ket{120} + \ket{201})
+ \gamma (\ket{021} + \ket{102} + \ket{210}).
\end{equation}


これで、この混合3キュートリット状態をアリス、ボブ、ジェラルドの間で分割できます。 左のキュートリットはアリスに属し、右のキュートリットはジェラルドに属しています。 彼らのキュートリットを考えると、彼らが所有する状態は$\ket{0},\ket{1}$および$\ket{2}$の等しい混合物を持っているので、誰も元の状態について何も知りません。
ただし、2人で秘密の状態$\ket{\phi_{secret}} $を再構築できます。 たとえば、アリスとボブは一緒になります。 アリスは自分のキュートリットをボブのmodulo 3に追加し、次にボブは自分の(新しい)キュートリットをアリスに追加します。 その結果は以下の状態です

\begin{equation}
\label{229}
(\alpha \ket{0} + \beta \ket{1} + \gamma \ket{2})
(\ket{00} + \ket{12} + \ket{21}).
\end{equation}

これを確認するために、$\alpha$の乗数だけを考えてみましょう。 アリスとボブが集まるとき、彼らは

\begin{equation}
\label{230}
\alpha (\ket{000} + \ket{111} + \ket{222}) + \cdots .
\end{equation}

アリスのキュートリットをボブのmodulo 3に追加すると、

\begin{equation}
\label{231}
\alpha (\ket{000} + \ket{111} + \ket{222}) + \cdots 
\to
\alpha (\ket{000} + \ket{121} + \ket{212}) + \cdots .
\end{equation}

次に、ボブの(新しい)キュートリットをアリスのキュートリットに追加します。

\begin{equation}
\label{232}
\alpha (\ket{000} + \ket{121} + \ket{212}) + \cdots 
\to
\alpha (\ket{000} + \ket{021} + \ket{012}) + \cdots 
\end{equation}

\begin{equation}
\label{233}
=
(\alpha \ket{0} + \cdots )(\ket{00} + \ket{12} + \ket{21}). 
\end{equation}



アリスのキュートリットは、他のキュートリットから解き放たれた秘密の状態$\ket{\phi_{secret}} $と同じになりました。 同様のプロセスによって、ジェラルドとボブは秘密の状態、またはアリスとジェラルドを回復することができます。
