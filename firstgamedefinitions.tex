\section{First game definitions and strategies:最初のゲームの定義と戦略}

前のセクションで暗示されているように、ゲーム$\Gamma$はセット$\Gamma=\Gamma$(プレーヤー、移動またはアクション、結果、利得)として定義できます。スピンフリップゲームでは、プレーヤーはアリスとボブであり、動きは行列$\sigma_x$または$\mathbf{1}$の適用であり、結果はスピン状態$u$または$d$であり、アリスへの利得は$+1$または$-1$でした。 最終状態がそれぞれ$d$か$u$かです。 これは$2$人のゼロサムゲームだったので、ボブへの見返りはアリスへの見返りとは正反対でした。

これまでのところ、ゲームの説明では、アリスとボブがどのように動きを決定したか、つまり、$\sigma_x$と$\mathbf{1}$のどちらをプレイするかをどのように決定したかについての説明は省略されています。
戦略は、ゲームの任意の段階で動きを決定するためのルールです。 つまり、この例では、移動は集合$\{ \mathbf{1}, \sigma_x \}$のメンバーですが、戦略はゲームの状態を一連の移動にマッピングする関数$f$です。$f \:$ゲームの状態$\rightarrow \{ \mathbf{1}, \sigma_x \}$ 。(量子ゲーム理論の文献では、この点について混乱しているようです。)「ゲームの状態」はプレイヤーに知られていない可能性があるため、これはあまり適切な定義ではありません。 プレイヤーは自分の動きしか知らないかもしれません。
それでは、これを次のように修正しましょう。アリスの戦略はマッピング$f_A$:$\{アリスの情報 \} \rightarrow \{アリスの動き\}$です。
そして、ボブも同様です。 スピンフリップゲームでは、アリスは電子の最初の準備の後、動きを選択する機会が1つしかないため、3つの動きのシーケンスの2番目、つまり中間のステップで1つの戦略を持ちます。 ボブは最初と最後のステップのための戦略を持っています。
したがって、一連の動きに関連するのは一連の戦略です。 経済学では、戦略はプレーヤーの情報に大きく依存しています。
特に興味深いのは非対称情報であり、あるプレーヤーが別のプレーヤーよりも情報の利点がある場合、またはプレーヤーの情報セットが同じではない場合です。
ボブがアリスができない量子の動きをすることができれば、明らかにボブは少なくともその点で情報の利点を持っています。
ゲームの許可された動きとペイオフを考えると、戦略はゲームに内生的であるため、戦略はゲームの定義の一部ではありません。 むしろ、ゲームを解くことは本質的にプレイヤーにとって最適な戦略を決定することを意味します。

情報セットの概念は重要です。 スピンフリップゲームでは、ボブもアリスも相手の動きを知ることができないと言いました。
この仮定を緩和したとしましょう。 そうすれば、アリスはボブの最初の動きを知り、それに応じて彼女の動きを選ぶことができますが、違いはありません。
ボブは、アリスの動きを見て(そして彼自身の最初の動きを知って)、電子をスピンアップ状態uのままにする最終的な動きをいつでも選択できました。 彼は100パーセントの確率で勝ちます。 それは「ゲーム」ではなく、ラケットです。したがって、この場合、そもそもゲームにするために、アリスとボブの情報セットを制限する必要があります。

ここで、例として、アリスとボブのそれぞれについて、次の戦略$f_A$と$f_B$を考えてみましょう。 これらは、何らかの確率メカニズムを使用した移動の選択を伴うため、混合戦略と呼ばれます。

\begin{equation}
\label{17}
f_A = play \ \mathbf{1} \ with \ probability  \ p = \frac{1}{2}, play \ \sigma_x \ with \ probablity \ q = \frac{1}{2}
\end{equation}


\begin{equation}
\label{18}
f_B = play \ \mathbf{1} \ with \ probability  \ p = \frac{1}{2}, play \ \sigma_x \ with \ probablity \ q = \frac{1}{2} .
\end{equation}

次に、表4の列を見ると、ボブが何をしても、アリスの期待されるペイオフ$\bar{\pi}_A$は常に

\begin{equation}
\label{19}
\bar{\pi}_A = \frac{1}{2}(+1) + \frac{1}{2}(-1)  =0
\end{equation}

一方、表4の行を見ると、ボブの期待される見返りは常に

\begin{equation}
\label{20}
\bar{\pi}_A = \frac{1}{4}(+1) + \frac{1}{4}(-1) + \frac{1}{4}(-1) + \frac{1}{4}(+1) =0
\end{equation}

もちろん、混合戦略と期待されるペイオフの概念が非常に理にかなっているためには、一連の$N$ゲームを検討する必要があります。

\begin{equation}
\label{21}
\Gamma_N \Gamma_{N-1} \Gamma_{N-2} \cdots \Gamma_{3}  \Gamma_{2} \Gamma_{1}  
\end{equation}

アリスへの実際のペイオフは、$x$を$N$ゲームの勝利数で表し、ペイオフセットのメンバーになります。

\begin{equation}
\label{22}
\Pi = \{f(x;N)\} = \{ 2x-N, for \ x = 0,1,\cdots ,N \}
\end{equation}

これらのペイオフの確率は

\begin{equation}
\label{23}
P(\Pi) = \{f(x;N,p)\} = \{ 
%\dbinom{n}{k} 
\left( \begin{array}{c}
N \\
x
\end{array} \right)
p^x q^{N-x}, for \ x = 0,1,\cdots ,N \}
\end{equation}

となります。

たとえば、$N = 3$の場合、アリスへの可能なペイオフは$\{-3,-1,1,3 \}$です。
そして、$p = 1$の場合、これらはそれぞれの確率$ \{ \frac{1}{8},\frac{3}{8},\frac{3}{8},\frac{1}{8} \} $です。
アリスの期待されるペイオフ$\bar{\pi}_A$は$0$ですが、$N$が奇数の場合、彼女の実際の
ペイオフが$0$になることはありません。

物理学者は、式(22)を、大規模な場合に考えられる結果の状態を与えるものとして認識します。
スピン$\frac{N \hbar}{2}$の粒子が測定されます。
この場合のスピンは、$(N+1)$状態の量子システムを定義します。
式(22)で与えられるスピン値(基本単位$\frac{\hbar}{2} $に関して)の可能な結果を伴う。 したがって、大規模な質量粒子の測定されたスピン状態は、アリスとボブの間の$N$スピンフリップゲームによって決定されると考えることができます。

表4に類似したペイオフのマトリックスで、一般的な2人のゼロサムゲームの場合、アリスの動きを混合戦略(動きに対する確率のセット)で、$P_ A = \{ a_1, a_2, \cdots , a_m \} $表すようにします。 ボブの混合戦略は $ P_B = \{ b_1, b_2, \cdots, b_n \} $ によって表されます。
アリスへのペイオフを$m \times m $ 行列 $ \pi_{ij} $ で表すとします。 次に、アリスへの期待される見返りは

\begin{equation}
\label{24}
\pi_A = \sum_{j=1}^{n} \sum_{i=1}^{m} \pi_{ij} a_i b_j.
\end{equation}

です。これに関連して、すべての有限の2人のゼロサムゲームについて、ミニマックス定理に言及する必要があります。

\begin{equation}
\label{25}
max_{P_A} ( min_{P_B} \bar{\pi}_A ) = min_{P_B} ( max_{P_A} \bar{\pi}_A ).
\end{equation}

つまり、アリスは予想されるペイオフを最大化するために予想される動きを選択し、ボブはアリスの予想されるペイオフを最小化するために予想される動きを選択します。 ミニマックス定理は、ボブの確率の最小化セットが与えられた場合のアリスの確率の最大化セットへの見返りは、アリスの確率の最大化セットが与えられた場合のボブの確率の最小化セットへの見返りに等しいと言えます。






