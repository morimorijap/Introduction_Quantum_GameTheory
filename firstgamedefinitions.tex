\section{First game definitions and strategies:最初のゲームの定義と戦略}

前のセクションで暗示されているように、あるゲーム$\Gamma$は集合$\Gamma=\Gamma$(プレーヤー、行動またはアクション、結果、利得)として定義できます。
スピンフリップゲームでは、プレーヤーはアリスとボブであり、行動は行列$\sigma_x$または$\mathbf{1}$の適用であり、結果はスピン状態$u$または$d$であり、アリスの利得は$+1$または$-1$でした。
最終状態がそれぞれ$d$か$u$かです。これは2人のゼロサムゲームだったので、ボブへの見返りはアリスへの見返りとは正反対でした。

これまでのところ、ゲームの説明ではアリスとボブがどのように動きを決定したか、つまり$\sigma_x$と$\mathbf{1}$のどちらを適用するかをどのように決定したかの説明は省略されています。
戦略は、ゲームの任意の段階で行動を決定するためのルールです。つまり、この例では行動は集合$\{ \mathbf{1}, \sigma_x \}$のある要素ですが、
戦略はゲームの状態を一連の行動にマッピングする関数$f: \text{ゲームの状態} \rightarrow \{ \mathbf{1}, \sigma_x \}$ です
(量子ゲーム理論の文献では、この点について不明瞭があるようです)。
「ゲームの状態」はプレイヤーに知られていない可能性があるためこれはあまり適切な定義ではありません。つまりあるプレイヤーは自分の行動以上のことをほとんど知らないかもしれません。
それでは、これを次のように修正しましょう。アリスの戦略はマッピング$f_A: \{\text{アリスの情報}\} \rightarrow \{\text{アリスの行動}\}$です。
そして、ボブも同様です。スピンフリップゲームでは、アリスは電子の最初の準備の後、行動を選択する機会が1つしかないため、3つの動きのシーケンスの2番目、つまり中間のステップで1つの戦略があります。
ボブは最初と最後のステップのための2つの戦略を持っています。
したがって、一連の行動に関連するのは一連の戦略です。経済学では、戦略はプレーヤーの情報に大きく依存しています。
特に興味深いのは非対称情報であり、あるプレーヤーが別のプレーヤーよりも情報の利点がある場合、またはプレーヤーの情報の集合が同じではない場合です。
もしボブがアリスができない量子操作ができれば、明らかにボブは少なくともその点で情報の利点があります。
ゲームの許可された動きとペイオフを考えると、戦略はゲームに内生的(endogenous)であるため、戦略はゲームの定義の一部ではありません。
むしろ、ゲームを解くことは本質的にプレイヤーにとって最適な戦略を決定することを意味します。

情報セットの概念は重要です。スピンフリップゲームでは、ボブもアリスも相手の行動を知ることができないと言いました。
この仮定を緩和したとしましょう。つまりアリスはボブの最初の行動を知り、それに応じて彼女の行動を選ぶことができますが、しかしそれは変化をもたらしません。
ボブは、アリスの行動が分かって(かつ彼自身の最初の行動は知っていて)、電子をスピンアップ状態$u$のままにする最終的な行動をいつでも選択できました。
彼は100パーセントの確率で勝ちます。それは「ゲーム」ではなくラケット(racket)です。したがって、この場合そもそもゲームにするために、アリスとボブの情報セットを制限する必要があります。

ここで例として、アリスとボブのそれぞれについて、次の戦略$f_A$と$f_B$を考えてみましょう。
これらは、何らかの確率メカニズムを使用した行動の選択を伴うため\emph{混合戦略}と呼ばれます。

\begin{align}
f_A = \text{play}\, \mathbf{1}\, \text{with probability}\, p = \frac{1}{2},
    \text{play}\, \sigma_x\, \text{with probablity}\, q = \frac{1}{2}
\end{align}

\begin{align}
f_B = \text{play}\, \mathbf{1}\, \text{with probability}\, p = \frac{1}{2},
    \text{play}\, \sigma_x\, \text{with probablity}\, q = \frac{1}{2}
\end{align}

次に、\autoref{tab:spin_flip:alice_payoff}の列を見ると、ボブが何をしてもアリスの期待されるペイオフ$\bar{\pi}_A$は常に次のようになります。

\begin{align}
\bar{\pi}_A = \frac{1}{2}(+1) + \frac{1}{2}(-1) = 0
\end{align}

一方、\autoref{tab:spin_flip:alice_payoff}の行を見ると、ボブの期待されるペイオフは常に次のようになります。

\begin{align}
\bar{\pi}_B = \frac{1}{4}(+1) + \frac{1}{4}(-1) + \frac{1}{4}(-1) + \frac{1}{4}(+1) = 0
\end{align}

もちろん、混合戦略と期待されるペイオフの概念が理にかなっているためには、連続した$N$個のゲームについて検討する必要があります。

\begin{align}
\Gamma_N \Gamma_{N-1} \Gamma_{N-2} \cdots \Gamma_{3} \Gamma_{2} \Gamma_{1}  
\end{align}

アリスへの実際のペイオフは、$x$が$N$個のゲームのうちの勝利数を表し、かつペイオフ集合の元(集合の要素)になります。

\begin{align}
\Pi = \{f(x;N)\} = \{ 2x - N, \text{for}\, x = 0,1,\dots,N \} \label{eq:alice_payoff_set}
\end{align}

これらのペイオフの確率は次のようになります。

\begin{align}
P(\Pi) = \{f(x;N, p)\} = \{ 
%\dbinom{n}{k} 
  \left(\begin{array}{c}
    N \\
    x
  \end{array}\right) p^x q^{N-x}, \text{for}\, x = 0,1,\dots ,N
\}
\end{align}

たとえば$N = 3$の場合、アリスの可能なペイオフは$\{-3,-1,1,3\}$です。
そして$p = 1$の場合、これらはそれぞれの確率$\{ \frac{1}{8},\frac{3}{8},\frac{3}{8},\frac{1}{8} \}$です。
アリスの期待されるペイオフ$\bar{\pi}_A$は$0$ですが、$N$が奇数の場合、彼女の実際のペイオフが$0$になることはありません。

物理学者は\autoref{eq:alice_payoff_set}を、スピン$\frac{N \hbar}{2}$の粒子の大規模に測定した場合に考えられる結果の状態を与えるものとして認識します。
この場合のスピンは\autoref{eq:alice_payoff_set}で与えられるスピン値の可能な結果(基本単位$\frac{\hbar}{2}$に関して)についての
$(N+1)$状態の量子システムを定義します。
したがって、大規模な粒子の測定されたスピン状態は、アリスとボブの間の$N$スピンフリップゲームによって決定されると考えてよいでしょう。

\autoref{tab:spin_flip:alice_payoff}に類似したペイオフのマトリックスで、一般的な2人のゼロサムゲームの場合、
アリスの行動を混合戦略(行動に対する確率の集合)で、$P_A = \{ a_1, a_2, \dots , a_m \}$表すようにします。
一方でボブの混合戦略は$P_B = \{ b_1, b_2, \dots, b_n \}$によって表されます。
アリスのペイオフを$m \times n$の行列$\lbrack\pi_{ij}\rbrack$で表すとします。次にアリスの期待されるペイオフは次のようになります。

\begin{align}
\pi_A = \sum_{j=1}^{n} \sum_{i=1}^{m} \pi_{ij} a_i b_j.
\end{align}

これに関連して、すべての有限の2人の次のゼロサムゲームについて言及している\emph{ミニマックス定理}について述べる必要があります。

\begin{align}
max_{P_A} ( min_{P_B} \bar{\pi}_A ) = min_{P_B} ( max_{P_A} \bar{\pi}_A ).
\end{align}

つまりアリスは期待できるペイオフを最大化するために可能な行動を選択し、
ボブはアリスの期待されるペイオフを最小化するために可能な行動を選択します。
ミニマックス定理は、ボブの確率の最小化セットが与えられた場合のアリスの確率を最大化する集合のペイオフは、
アリスの確率の最大化セットが与えられた場合のボブの確率を最小化する集合のペイオフに等しいと言えます。
