\section{Amplitudes and superpositions and his cheatin' heart:振幅と重ね合わせと彼の心をだます}

次の形式の量子状態(ベクトル)$\psi$を考えてみましょう。ここで、$a$と$b$は複素数である可能性があります。

\begin{align}
\psi = au + bd
\end{align}

量子計算では、この重ね合わされた2次元状態は\emph{キュービット(qubit)}と呼ばれ、後で詳しく説明します。
ここで、$a$と$b$は振幅であり、$\psi$(フォンノイマン)測定は確率$|a|^2$で基本状態$u$を取得し、
もう一方で測定により確率$|b|^2$で基本状態$d$を取得します。ここで、$|a|^2 + |b|^2 = 1$です。
(複素数$a$とその複素共役$a^{\ast}$の場合、$aa^{\ast} = a^{\ast} a = |a|^2$であることを思い出してください。)

これにより、古典的なアナログが存在しないスピンフリップゲームのバリエーションを含む複数のゲームの可能性が高まります。
たとえば、$a = b = \frac{1}{\sqrt{2}}$に設定します。
その場合、$u$または$d$のいずれかの確率は$\left|\frac{1}{\sqrt{2}}\right|^2 = \frac{1}{2}$です。
したがって、混合戦略がボブまたはアリスのどちらによって選択されたかに関係なく、確率は状態ベクトルの測定値となります。

ここで、$u$と$d$は正規直交です(つまり、$u$と$d$の内積は$0$であり、$u$または$d$とそれ自体の内積は$1$です)。
したがって、内積として次の$a$を取得できます。 

\begin{align}
\braket{\psi, u} = a \braket{u, u} + b \braket{d, u} = a(1) +b(0) =  a
\end{align}

同様の計算で$b$が得られます。

\subsection{アリスの不正} %「アリスのいかさま」とGoogle翻訳ではでる。

ここで、スピングリップゲームのバリエーションを考えてみましょう。これを\emph{アリスチート}と呼びます。
ここで、アリスは電子のスピン状態の初期準備で不正行為をする方法があります。
まず、ボブが電子をスピン状態$u$にすると考えていることを知って、彼女が最初にスピン状態$d$の電子を準備するとします。
それ以外の点ではゲームは以前とまったく同じです。ボブとアリスの両方が$\textbf{1}$または$\sigma_x$のいずれかを適用します。
スピン状態の配置が\autoref{tab:spin_flip:spin_seq}で変化し、アリスへのペイオフの配置が\autoref{tab:spin_flip:alice_payoff}で変化することは簡単にわかりますが、ペイオフのセット$\Pi$は同じであり、対応するペイオフ確率$P(\Pi)$アリスへは変わりません。したがって、アリスの不正は無駄になりました。
彼女は初期状態を$u$から$d$に変更しただけで、ゲームの結果に影響はありませんでした。以前は$+1$でしたが、現在は$-1$でその逆も同様です。

それで、アリスは何か他のことを試みます。 彼女は初期状態を$\frac{1}{\sqrt{2}}(u + d)$に選択します。
次に、ボブが$1$をプレイするか$\sigma_x$をプレイするかにかかわらず、彼の動きはゲームの状態を変更しません。

\begin{align}
\textbf{1} \left[\frac{1}{\sqrt{2}}(u + d)\right]
= \frac{1}{\sqrt{2}}(\textbf{1}u + \textbf{1}d)
= \frac{1}{\sqrt{2}}(u + d) \label{eq:alice_cheat1}
\end{align}

\begin{align}
\sigma_x \left[ \frac{1}{\sqrt{2}}(u + d)\right]
= \frac{1}{\sqrt{2}}(\sigma_x u + \sigma_x d)
= \frac{1}{\sqrt{2}}(d + u)
\end{align}

$u + d = d + u$なので、$1$または$\sigma_x$のいずれかをしても状態は変化しません。
しかし、電子の(変更されていない)状態の最終測定が行われると、アリスは、最終的な重ね合わせ状態の測定が等しい確率で$u$または$d$を生成するため、
また同じ確率でドルを勝ち取るか失うことに不満を感じます。
単一のゲームの場合、ペイオフセット$\Pi$と対応する確率$P(\Pi)$は次のとおりです。

\begin{align}
\Pi = \{-1, +1\}
\end{align}

\begin{align}
P(\Pi) = \left\{ \left( \frac{1}{\sqrt{2}} \right)^2, \left( \frac{1}{\sqrt{2}} \right)^2 \right\} 
= \left\{ \frac{1}{2}, \frac{1}{2} \right\} \label{eq:alice_cheat2}
\end{align}

\subsection{ボブの不正}

基本的なスピンフリップゲームに戻りましょう。ここでは、悔い改めたアリスが初期の$u$状態で電子を準備し、混合戦略に従うという詳細を追加し、彼女は確率$p = 1$でそれぞれ$\textbf{1}$または$\sigma_x$を選択します。
しかし今、私たちはボブにチートを許可します。ボブは初期電子状態を準備しないため、ボブの不正行為の方法はアリスの方法とは異なります。
ボブはどんな卑劣なことをすることができるでしょうか?ボブは、袖の上にいくつかの余分なパウリスピン行列、つまり$\sigma_y$と$\sigma_z$、およびこれらの線形結合を持っています。
さらに、ボブは最後の行動をします。ボブがいわゆるアダマール演算子$H = \frac{1}{\sqrt{2}}(\sigma_x + \sigma_x)$を適用するとします。

\begin{align}
H = \frac{1}{\sqrt{2}}
\begin{pmatrix}
1 & 1 \\
1 & -1 \\
\end{pmatrix}
\end{align}

ボブの最初の行動の後、スピン状態は次のようになります。

\begin{align}
Hu = \frac{1}{\sqrt{2}}
\begin{pmatrix}
1 & 1 \\
1 & -1 \\
\end{pmatrix}
\begin{pmatrix}
1  \\
0  \\
\end{pmatrix}
=
\frac{1}{\sqrt{2}}
\begin{pmatrix}
1  \\
1  \\
\end{pmatrix}
=
\frac{1}{\sqrt{2}}(u + d)
\end{align}

\autoref{eq:alice_cheat1}--\ref{eq:alice_cheat2}で見たように、アリスの混合戦略はこの状態を変更しません。
それからボブはもう一度$H$を適用します。

\begin{align}
H(Hu) = \frac{1}{\sqrt{2}}
\begin{pmatrix}
1 & 1 \\
1 & -1 \\
\end{pmatrix}
\frac{1}{\sqrt{2}}
\begin{pmatrix}
1  \\
1  \\
\end{pmatrix}
=
\frac{1}{2}
\begin{pmatrix}
2  \\
0  \\
\end{pmatrix}
= u
\end{align}

ボブは常に勝ちます。これは、状態の\emph{重ね合わせ}を作成するボブの能力(および彼は最後の行動ができること)に起因します。
ボブがアダマール行列$H$を$u$に適用した後、生死が同時なSchrödingerの猫のように、電子スピンは同時に$u$と$d$の両方になります。
アリスは、確率$p$で$\textbf{1}$を適用し、確率$1-p$で$\sigma_x$を適用するという古典的な混合戦略を行っても結果を変更することはできません。
