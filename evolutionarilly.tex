\section{Evolutionarily stable strategy game: 進化的に安定した戦略ゲーム}

量子ゲームは私たちの周りで分子レベルで毎日行われているようです。 GogoneaとMerz (参考文献26)は、タンパク質の折り畳みにおいて量子力学的レベルでゲームが行われていることを示しています。 ターナーとチャオ(参考文献67)は、RNAファージ内のウイルス間の競合的相互作用の進化を研究し、ファージの適応度が2人の囚人のジレンマゲームに準拠したペイオフマトリックスを生成することを発見しました。 生物学のゲーム理論の側面について簡単に触れたいと思います。

ナッシュ均衡の概念に関連して以前に定義した進化的安定戦略(ESS)の概念は、集団生物学のいくつかの問題に対処するためにゲーム理論(参考文献64)に導入され、複数のナッシュ均衡が存在する可能性があります。
%%% のちほど書籍を参考に訳をする
進化論とゲーム理論(参考文献44)で、メイナードスミスは、「ゲーム理論は、それが最初に設計された経済行動の分野よりも生物学に容易に適用される」と述べた。

対称的なバイマトリックス(つまり、2×2)ゲームをプレイするために、ペアでランダムに一致するN人のメンバーの母集団を考えてみます。

対称とは、次のことを意味します。 Sをプレーヤーの動きのセットとし、$s_i$、$s_j$をアリスとボブの両方が利用できる動きとします。

次に、彼女が$s_i$をプレイしてボブが$s_j$をプレイしたときのアリスの期待されるペイオフは、ボブが$s_i$をプレイしてアリスが$s_j$をプレイした場合のボブの期待されるペイオフと同じです。

\begin{equation}
\label{172}
\bar{\pi}_A (s_i, s_j) = \bar{\pi}_B (s_j, s_i)
\end{equation}

つまり、アリスのペイオフマトリックス$\Pi_A$は、ボブのペイオフマトリックスの転置です:$\Pi_A = \Pi_B^T $。 これは、ゲームの対称性を定義します。 時間の経過とともに高いペイオフで$s_i$が移動すると、ゲームは進化的になり、徐々にそれらの$s_j$を低いペイオフに置き換えます。 そのようなゲームで、メイナード・スミスとプライス(参考文献43)は、ESSを採用する集団が小さな侵入グループに耐えることができることを示しました。

しかし、古典的な動きをしている間に平衡状態にある現在の人口が、量子の動きをしている人口によって侵略された場合はどうなるでしょうか? これはIqbalとToorによって検討された問題です(参考文献33)。
対称バイマトリックスゲームでムーブ$s_i$をプレイする母集団の割合が$p_i$であり、ムーブ$s_j$をプレイする割合が$p_j$であると仮定します。 動き$s_i$と$s_j$の適合度$w$を次のように定義します。

\begin{equation}
\label{173}
w(s_i) =
p_i \bar{\pi} (s_i, s_j) + p_j \bar{\pi} (s_i, s_j)
\end{equation}

\begin{equation}
\label{174}
w(s_j) =
p_i \bar{\pi} (s_j, s_i) + p_j \bar{\pi} (s_j, s_j)  
\end{equation}

最初の方程式は、$move \  s_i$の適合性は、$s_i$をプレイしている対戦相手に対して$s_i$をプレイした場合のペイオフと、$s_j$をプレイしている対戦相手に対して$s_i$をプレイした場合のペイオフの加重平均であることを示しています。 それぞれの重みは、$s_i$と$s_j$をプレイしている母集団の比率です。 2番目の式は、インデックスが切り替えられた最初の式と実際には同じです。

量子進化的に安定した戦略ゲームでは、2つの集団グループ間でプレイされる対称バイマトリックスゲームが囚人のジレンマゲームであると想定します。 このゲームのペイオフマトリックスは、前に表VIに示したものです。 一方のプレーヤーのペイオフマトリックスは、もう一方のプレーヤーのペイオフマトリックスの転置であり、対称性に必要であることに注意してください。量子囚人のジレンマゲームで使用されるユニタリ行列 $ U = \frac{1}{\sqrt{2}}(\mathbf{1}^{\otimes 2}  + i \sigma_x^{\otimes 2} )  $も2人のプレーヤー間で対称であることに注意してください。

古典的な動きの場合、ペイオフ状態$ {sA, sB} = {D,D}$および${\pi(s_A), \pi(s_B)} = {1,1}$は、ナッシュ均衡であり、進化的に安定した戦略でもあります。 ただし、量子運動を行う変異体の侵入力の影響を考慮してください。

参照しやすいように、ここでは表VIIIを表XIIとして再現します。 ${\mathbf{1}, \sigma_x}$を古典としてラベル付けします。
移動し、ミュータントが移動すると${H, \sigma_z}$になります。

\begin{table}[htb]
\caption{$\mathbf{1}$の古典的な動きをしている集団、$\sigma_x$は、量子の動きHをしている突然変異体によって侵略されています。 後の変異体の侵入は$\sigma_z$を再生し、前の変異体を一掃します。}
\centering
\begin{tabular}{|r|r|r|r|r|} \hline
 & Classical $\mathbf{1}$ & Classical $\sigma_x$ & Mutant $H$ & Mutant $\sigma_z$ \\ \hline
Classical $\mathbf{1}$ & $(3,3)$ &  $(0,5)$ &  $(\frac{1}{2},3)$ & $(1,1)$  \\
Classical $\sigma_x$ &  $(5,0)$ &  $(1,1)$ &  $(\frac{1}{2},3)$ & $(0,5)$  \\
Mutant $H$  & $(3,\frac{1}{2})$ &  $(3,\frac{1}{2})$ &  $(2 \frac{1}{4},2 \frac{2}{4})$ & $(1 \frac{1}{2},4)$  \\
Mutant $\sigma_z$ & $(1,1)$ &  $(5,0)$ &  $(4, 1 \frac{1}{2})$ & $(3,3)$  \\   \hline
\end{tabular}
\end{table} 
% 表XII:$\mathbf{1}$の古典的な動きをしている集団、$\sigma_x$は、量子の動きHをしている突然変異体によって侵略されています。 後の変異体の侵入は$sigma_z$を再生し、前の変異体を一掃します。\\

$\sigma_x$は$H$に対して進化的に安定していないことがわかります。$\sigma_x$をプレイしているメンバーは消滅し、人口はまもなく$H$をプレイしている変異体で構成されます。
新しいESSは、どちらかのミュータントパーティに$2 \frac{1}{4}$のペイオフをもたらします。
この新しい集団が$\sigma_z$を再生するさまざまな変異体によって侵略された場合、$H$はもはやESSではありません。
$H$をプレイしているメンバーは消滅し、人口はまもなく$\sigma_z$をプレイしているミュータントで構成されます。
これらの変異体は3の見返りを享受し、元の集団と比較すると肥えて幸せに見えます。

