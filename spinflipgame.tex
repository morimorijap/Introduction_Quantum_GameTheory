\section{The spin flip game:スピンフリップゲーム}
\label{sec:the_spin_flip_game}

電子には2つのスピン状態があり、スピンアップとスピンダウンです。
アリスとボブの間で行われる電子スピンフリップの簡単なゲームを考えてみましょう。
アリスは最初にスピンアップ状態uの電子を準備します。この最初のステップの後、ボブは$\sigma_x$または$\mathbf{1}$行列のいずれかを$u$に適用し、次のいずれかになります。

\begin{align}
\sigma_x u = d \, \text{or} \, \mathbf{1}u = u.
\end{align}

次に、アリス(ボブの行動や電子の状態を知らない)が順番を取り、電子スピンに$\sigma_x$または$\mathbf{1}$のいずれかを適用します。
次に、ボブ(アリスの行動や電子の状態を知らない)の順番となります。最後に電子のスピン状態が測定されます。
$u$状態の場合、ボブは1ドルを獲得し、アリスは1ドルを失います。$d$状態の場合、アリスは1ドルを獲得し、ボブは同じ金額を失います。

ボブ(列)とアリス(行)による可能な選択肢のシーケンスを表\ref{tab:spin_flip:player_seq}に要約します。
アリスの動きは、右から左に読んだ3つのシーケンスの真ん中であることに注意してください。

\begin{table}[H]
\caption{プレイヤーの動きのシーケンス}
\label{tab:spin_flip:player_seq}
\centering
  \begin{tabular}{|r||r|r|r|r|} \hline
    $Alice \backslash Bob$ & $\mathbf{1}, \mathbf{1}$  &  $\mathbf{1}, \sigma_x $  & $ \sigma_x,\mathbf{1} $ & $\sigma_x, \sigma_x $  \\ \hline \hline
    $\mathbf{1}$ & $\mathbf{1}, \mathbf{1},\mathbf{1}$ & $\mathbf{1}, \mathbf{1},\sigma_x$ & $\sigma_x, \mathbf{1},\mathbf{1}$ & $\sigma_x, \mathbf{1},\sigma_x$ \\
    $\sigma_x$ & $\mathbf{1}, \sigma_x,\mathbf{1}$ & $ \mathbf{1}, \sigma_x,\sigma_x$ & $\sigma_x, \sigma_x,\mathbf{1}$ & $\sigma_x, \sigma_x,\sigma_x$\\ \hline
  \end{tabular}
\end{table}

たとえば、$\mathbf{1},\mathbf{1}, \sigma_x$は、ボブが$\sigma_x$を適用し、次にアリスが$\mathbf{1}$を適用し、次にボブが$\mathbf{1}$を適用したことを意味します。
最終的な結果は$\mathbf{11} \sigma_x u = d.$です。
したがって、アリスは1ドルを獲得します。
初期の$u$状態から各移動後のスピン状態の配列は、表\ref{tab:spin_flip:spin_seq}に示されています。繰り返しますが、3つの各シーケンスは右から左に読む必要があります。

\begin{table}[H]
\caption{スピン状態のシーケンス}
\label{tab:spin_flip:spin_seq}
\centering
  \begin{tabular}{|r||r|r|r|r|} \hline
    $Alice \backslash Bob$ & $\mathbf{1}, \mathbf{1}$  &  $\mathbf{1}, \sigma_x $  & $ \sigma_x,\mathbf{1} $ & $\sigma_x, \sigma_x $  \\ \hline \hline
    $\mathbf{1}$ & $u,u,u$ & $d,d,d$ & $d,u,u$ & $u,d,d$ \\
    $\sigma_x$   & $d,d,u$ & $u,u,d$ & $u,d,u$ & $d,u,d$ \\ \hline
  \end{tabular}
\end{table}

最後に、表\ref{tab:spin_flip:alice_payoff}は、アリスへのペイオフ(利得)を示しています。
最終スピンが$d$状態の場合は正であり、$u$状態であれば負です。

\begin{table}[H]
\caption{アリスのペイオフ}
\label{tab:spin_flip:alice_payoff}
\centering
  \begin{tabular}{|r||r|r|r|r|} \hline
    $Alice \backslash Bob$ & $\mathbf{1}, \mathbf{1}$  &  $\mathbf{1}, \sigma_x $  & $ \sigma_x,\mathbf{1} $ & $\sigma_x, \sigma_x $  \\ \hline \hline
    $\mathbf{1}$ & $-1$ & $+1$ & $+1$ & $-1$ \\
    $\sigma_x$   & $+1$ & $-1$ & $-1$ & $+1$ \\ \hline
  \end{tabular}
\end{table}


これは基本的なスピンフリップゲームであり、これを2つの方向に拡張していきます。
1つは確率的な動きを考慮すること、もう1つは状態の量子重ね合わせ(量子もつれなし)を考慮することです。
ただし、これを行う前にいくつかの基本的なゲーム理論の用語について考えてみましょう。
