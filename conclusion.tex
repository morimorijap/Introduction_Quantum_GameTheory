\section{Conclusion}

この時点で、読者は量子ゲーム理論を始めるのに十分な背景を取得しました。 もちろん、参考文献が示すように、言うべきことはもっとたくさんあります。 読者は特に量子計算に関する注記(参考文献21, 45, 61)を参照してください。

このエッセイは、伝統的なゲーム理論が量子ゲーム理論のサブセットであり、後者ははるかに豊富な構造と幅広い結果のセットを持っていることを示しています。
これが、量子ゲーム理論を実行するために必要なことすべての正当化です。 何も諦めず、後者に切り替えることでより多くが得られます。
したがって、伝統的なゲーム理論の研究は、進化的に安定した戦略でもナッシュ均衡でもなく、絶滅した種のコミ箱と非均衡の見返りに委ねられます。
そうは言っても、たとえるならば、量子ゲーム理論の現状は、突然変異体の侵入に耐えられるのだろうか?
それらの侵入する突然変異体が、量子力学の何が悪いのかを修正するためにやってくる数理経済学者になることを願っています。
確かに、ランベルティーニ(参考文献37)は、数理経済学と量子力学は同型であると主張しています。

量子ゲーム$\Sigma = \Sigma(\mathbf{H}, \Lambda, U, \{ s_i \}_j, \{ \pi_i \}_j  ) $ここで$\mathbf{H}$はヒルベルト空間です。
  $\Lambda$はゲームの初期状態です。
  $U$は、ゲームの開始時と終了時にすべてのプレーヤーのキュービットに適用されるユニタリ行列です。
$ \{ s_i \}_j$は、凸結合を含む、プレーヤー$j$の一連の動きです。 および$ \{ \pi_i \}_j$  は、プレーヤー$j$へのペイオフのセットです。
ゲームの目的は、プレーヤーjの期待される見返りを最大化する戦略を内生的に決定することです。
一般に、純粋な量子移動$s_i$は、プレーヤーの個々のキュービットに適用されるユニタリ行列です。

このエッセイの過程で、「The spin flip game:スピンフリップゲーム」、
「Guess a number games:ナンバーゲームを推測する」IおよびII、RSAゲーム、囚人のジレンマ、男女の戦い、Newcombのゲーム、進化的に安定した戦略ゲーム、コインフリップゲーム、 疑似テレパシーゲームおよびテレポーテーション、「 Quantum secret sharing量子秘密共有」、状態推定、および量子クローニングのゲーム理論的側面を述べた。 スピンフリップゲームでは、ボブはアダマール変換Hを介して量子重ね合わせを利用して常にゲームに勝つことができましたが、この結果はプレーヤーの動きのシーケンスにも依存することを確認しました。 ナンバーゲームを推測するための鍵は、グローバー検索アルゴリズムを使用して、ヒルベルト空間の状態ベクトルを未知の数のおおよその位置に回転させることでした。
この検索は、重ね合わせと$f_a$ オラクルの呼び出しを使用して、$N$回の移動から$\sqrt{N}$回の移動に高速化されました。
ナンバーゲームを推測するIIでは、Bernstein-Vaziraniオラクルを使用して、オラクルを$1$回呼び出した後、不明な番号のWalsh変換$W_{2^n}$を作成しました。
RSAゲームでは、Shorの因数分解アルゴリズムを使用して、整数の重ね合わせた状態を、与えられた複合RSA素数 $N = pq$に対して$\frac{2^{2n}}{r}$の整数倍に投影しました。ここで、$r$はテストされた要素の次数です。
確率は、量子フーリエ変換を使用して制御されました。

囚人のジレンマゲームでは、量子の追加により$H$と$\sigma_z$が1に移動し、$\sigma_x$が従来のゲームの結果に追加され、実際にナッシュ均衡としてパレート最適点に到達することがわかりました。
男女の戦いのゲームでは、同じ量子の動きが、純粋な戦略でユニークなナッシュとパレートの最適な均衡を生み出しました。混合戦略におけるアリスとボブの平等、ナッシュ均衡とパレート最適を説明しました。
Newcombのパラドックスは、スーペリアビーイングの全知に取って代わった重ね合わせの使用を通じて、アリスの選択を完全に予測(制御)するスーペリアビーイングの能力と、アリスの側で不正行為をするインセンティブによって解決されました。
これらのゲームは、ユニタリ行列$U$を使用することにより、「協調的」と「非協調的」のカテゴリの部分的な無関係性も示しています。
プレーヤーのキュービットがゲームに巻き込まれている場合、プレーヤーが自分の期待効用を最大化することに単に焦点を合わせると、コミュニケーションの隠されたチャネル(見えざる手)があります。
進化的に安定した戦略ゲームでは、量子の動きをしている侵入変異体は、古典的な動きだけをしている既存の種を一掃することができました。
コイントスゲームは、エンタングルメントのないゲームで、量子オラクルを使用して不公平なゲームを公正なゲームに変えることを実証しました。

疑似テレパシーゲームでは、量子もつれ状態を共有している限り、プレイヤーが共謀してゲームに勝つために、プレイヤー間のコミュニケーションは必要ありませんでした。
Nの適切なサブセットは、$0$のペイオフを期待していましたが、ゲームはN人のプレーヤー全員の暗黙の連立で確実に勝つことができました。また、$N$次元の確率空間が疑似テレパシーゲームのコアであることがわかりました。
これは、量子もつれが量子確率を引き起こすことを意味しますか?キュービット状態は観測不能であり、測定中は測定ベース(通常は$0$または$1$)に投影されるため、破壊されることがわかりました。
これは機会と困難を生み出します。ベルベースでの測定は、テレポーテーションプロトコルの中心です。
量子状態は特定の忠実度でのみ複製できますが、秘密分散と安全な通信に使用できます。
ベイズフレームワークで最尤法を使用した量子状態識別の問題、または密度行列のブロッホ球表現に関連して同じものを使用した量子状態推定の問題は、経済学者にとって基本的に異質な概念ではありません。

PiotrowskiとSladkowski (参考文献59)は、彼らがQuantum anthropic principle(量子人間原理)と呼んでいるものを述べています。文明市場の初期段階で古典的な法則が支配されていたとしても、比較優位を伝える際の量子アルゴリズムの比類のない効率は、量子 振る舞いは古典的なものよりも優先されます。 自然はすでに量子ゲームをしているので、人間は自分の量子コンピューター(人間の脳)も使ってそうしているように見えます。
したがって、投機的ではありますが、Complexity DigestでのGottfried Mayerのコメントは、「人間の決定が微視的な量子イベントにたどることができれば、自然は進化する複雑な脳で量子計算を利用したであろうと予想されるでしょう。
  その意味で、量子コンピューターは量子ルールに従って市場ゲームをプレイしていると言うことができます。(参考文献42)」とのこと。これについて、今のところハッキリしていません。
  
