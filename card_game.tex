\section{Card game: a quantum game without entanglement カードゲーム:エンタングルメントのない量子ゲーム}

次のゲームはエンタングルメントを使用していませんが、数学的な設定ではヒューリスティックであり、後続のより複雑なゲームの準備として適しています。 ボブとアリスは次のカードゲームをプレイします(参考文献17)。
次のマークを除いて、3枚のカードがあります。それ以外は同じです。最初のカードの両側に円があります。 2枚目のカードには両側にドットがあります。 3枚目のカードは片面に円、もう片面にドットがあります。 アリスは3枚のカードをブラックボックスに入れ、それを振って3枚のカードをランダム化します。
ボブは箱から盲目的に1枚のカードを引くことができます。 両側に同じマークがある場合、アリスはボブから+1を獲得します。 カードの両面に異なるマークが付いている場合、ボブはアリスから+1を獲得します。 
もちろん、両側に同じマークが付いている2枚のカードであるアリスはペイオフ$ \bar{\pi}_A = \frac{2}{3} (1) + \frac{1}{3}(-1) = \frac{1}{3} $を期待し、ボブはペイオフ$\bar{pi}_B = \frac{1}{3} (1) + \frac{2}{3} (-1) = - \frac{1}{3} $を期待しています 。 ゲームはボブにとって不公平です。

古典的な意味でゲームを公平にする1つの方法は、ボブがブラックボックスを見て、3枚のカードの1枚を引く前に上面を見ることができるようにすることです。
次に、ボブが3枚のカードの中で2つの円が上を向いているのを見た場合、彼はそれら2つのカードの1つをランダムに描きます。一方、2つのドットが上を向いているのを見た場合、彼は後者の2つのカードの1つをめったに描きません。
同じ上向きのマークが付いた2枚のカードの1つは、それぞれの面に異なるマークがなければならないので、これはボブに期待されるペイオフ$\pi_B= 0$を与えます。
ゲームは今や公平になるでしょう。 ただし、ボブにこれを行わせるつもりはありません。 実際、それは彼が中を見ることができないようにブラックボックスですが、彼は手を差し込んで1枚のカードを引き出すことができます。

代わりに、3つのカードすべての上面を見るのと同等の量子を作成するために、1)ボブがブラックボックスまたはキュービットデータベースに単一のクエリを実行できるようにします$\ket{r}$; そして、2)ボブが引いたカードの上面を見て、ゲームから撤退できるようにします。
この設定は非常に人工的であり、同じゲームについて説明していることすら疑わしいですが、この量子化されたバージョンのカードゲームでは、いくつかのヒューリスティックなポイントを作成できます。

\begin{equation}
\label{175}
\ket{r} = \ket{r_0 r_1 r_2}
\end{equation}

ここで、$r_k \in \{0,1 \}$です。

ボブのクエリの一部として、次のユニタリ行列$U_k$が必要になります。

\begin{equation}
\label{176}
U_k = \left(  
\begin{array}{cc}
 1  & 0\\
 0  & e^{i \pi r_k}
\end{array}
\right).
\end{equation}

$rk = 0$の場合、$U_k = \mathbf{1}$であり、$r_k = 1$の場合、$U_k = \sigma_z$であることに注意してください。 ここで、アダマール行列$H$を$U_k$に適用して、$H U_k H$を形成し、次の式を取得します。

\begin{equation}
\label{177}
H  U_k  H = \frac{1}{2} \left(  
\begin{array}{cc}
 1  & 1\\
 1  & -1
\end{array}
\right)
\left(  
\begin{array}{cc}
 1  & 0\\
 0   & e^{i \pi r_k}
\end{array}
\right)
\left(  
\begin{array}{cc}
 1  & 1\\
 1   & -1
\end{array}
\right)
=
\frac{1}{2}
\left(  
\begin{array}{cc}
 1+e^{i \pi r_k}  & 1-e^{i \pi r_k}\\
 1-e^{i \pi r_k}  & 1+ e^{i \pi r_k}
\end{array}
\right).
\end{equation}

したがって、この変換を状態|0⟩に適用すると、次のようになります。


\begin{equation}
\label{178}
H  U_k  H \ket{0} 
= 
\frac{1}{2}
\left(  
\begin{array}{cc}
 1+e^{i \pi r_k}  & 1-e^{i \pi r_k}\\
 1-e^{i \pi r_k}  & 1+ e^{i \pi r_k}
\end{array}
\right)
\left(  
\begin{array}{c}
 1  \\
 0
\end{array}
\right)
=
\frac{1}{2}
\left(  
\begin{array}{cc}
 1+e^{i \pi r_k} \\
 1-e^{i \pi r_k}  
\end{array}
\right)
=
\frac{1 + e^{i \pi r_k}}{2} \ket{0}
+
\frac{1 - e^{i \pi r_k}}{2} \ket{1}.
\end{equation}

$rk = 0$の場合、$H U_k H \ket{0} = \ket{0}$であるのに対し、$r_k = 1$の場合、$H U_k H \ket{0}= \ket{1}$であることに注意してください。 したがって、

\begin{equation}
\label{179}
H U_k H \ket{0} = \ket{r_k}.
\end{equation}

それでは、ボブがブラックボックスの状態$\ket{r}$に依存するクエリマシンを持っていると仮定しましょう。 マシンには3つの入力があり、3つの出力を提供します。 3枚のカードのアップサイドマークを決定するために、ボブは$\ket{000}$を入力して以下を取得します。

\begin{equation}
\label{180}
(H U_k H \otimes H U_k H \otimes  H U_k H ) \ket{000} = \ket{r_0 r_1 r_2}.
\end{equation}

したがって、ボブの質問の後、彼は3枚のカードの逆さまのマークを知っています。セット$S0 = \{ 3-qubitsの順列 \{ \ket{0}, \ket{0},\ket{1} \} \}$のある要素またはセット$S1 = \{ 3-qubitsの順列 \{ \ket{0}, \ket{1},\ket{1} \} \}$のある要素のいずれかです。

$S_0$がブラックボックスの状態を説明している場合、ボブは勝ったカードの表側に円があることを知っています。
$S_1$がブラックボックスの状態を表す場合、ボブは勝ったカードが上向きの面にドットを持っていることを知っています。
だから今、ボブは自分のカードを引き、表向きの顔だけを見るようになります。 描かれたカードの表側に円があり、$black \  box \in S_0$の場合、ボブは同じ確率で勝ちます。
しかし、$black \  box \in S_1$の場合、ボブは、引いたカードが負けたカードであることを知っているため、プレーを拒否します。
描かれたカードの表側にドットがある場合も同様の分析が当てはまります。

したがって、データベースへのクエリは、ブラックボックスに表向きに表示されている2つの円または2つのドットがあるかどうかをボブに示します。したがって、彼がカードを引くと、2つの逆さまのマークと一致する場合、50-50のチャンスがあることがわかります。 引き分けたカードが2つの上向きのマークと一致しない場合、そのカードは間違いなく敗者であり、彼はゲームから撤退するオプションを行使する必要があります。

エンタングルメントに関しては、演算子$H$と$U_k$はキュービットの単純な線形結合を形成しますが、量子クエリマシンはこれらの演算のテンソル積です。
したがって、このゲームには状態の絡み合いはありません。
Du et. al によれば、一般的なルールは、静的量子ゲームでは古典的な結果との違いを生み出すためにエンタングルメントが必要であるように見えるが、動的ゲームではそうではないことに注意すると記述しています。
  重要なのは、他のキュービットの状態に影響を与えるプレーヤーの能力です。
これは、エンタングルメントまたは動的ゲームのタイムステップを通じて実行できます。
